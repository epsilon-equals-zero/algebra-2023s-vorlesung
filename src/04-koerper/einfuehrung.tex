\section{Einführung}

\begin{definition}
    Sei $L$ ein Körper. Wir nennen $K \subseteq L$ einen \emph{Unterköper}\index{Unterkörper}, wenn $1 \in K$ und $K$ ein Körper ist. Dafür schreiben wir auch $K \leq L$. In diesem Kontext heißt $L$ auch \emph{Oberkörper}\index{Oberkörper} von $K$.

    Wir nennen
    $$ \bigcap \{ U \leq L \mid U \text{ Unterkörper von } L \} $$
    den \emph{Primkörper}\index{Primkörper} von L.

    Sei $K \leq L, S \subseteq L$ so definieren wir die \emph{Körpererweiterung von $K$ um $S$}\index{Körpererweiterung} durch
    $$ K(S) := \bigcap \{ U \mid K \leq U \leq L, U \supseteq S \}. $$
    Ist $S = \{ \alpha_1, \hdots, \alpha_n \}$, so schreiben wir auch $K(\alpha_1, \hdots, \alpha_n)$.
    
    Im Gegensatz dazu ist die \emph{Ringerweiterung von $K$ um $S$}\index{Ringerweiterung} definiert:
    $$ K[S] := \bigcap \{ U \mid U \;\mathrm{Ring}\land K \subseteq U \subseteq L, U \supseteq S \}. $$
    Ist $S = \{ \alpha_1, \hdots, \alpha_n \}$, so schreiben wir auch $K[\alpha_1, \hdots, \alpha_n]$.
\end{definition}

\begin{remark}
    Beispielsweise gilt $\mathbb{R}(i) = \mathbb{C}$, wie wir später noch sehen werden.
\end{remark}

\begin{remark}
    Sei $K$ ein Körper, dann ist $K$ ein Unterkörper des Quotientenkörpers $Q$ von $K[x]$. Also ist $K \leq Q$ eine Körpererweiterung.
\end{remark}

\begin{definition}
    Ein Körper $K$ heißt \emph{Primkörper}\index{Primkörper}\index{Köper!Prim-}, wenn $K$ keine echten Unterkörper hat.
\end{definition}

\begin{theorem}
    Sei $K$ ein Primkörper.
    \begin{itemize}
        \item Ist $\chara K = 0$, so ist $K \cong \mathbb{Q}$.
        \item Ist $\chara K = p \in \mathbb{P}$, so ist $K \cong \mathbb{Z}_p$.
    \end{itemize}
\end{theorem}

\begin{proof}
    Wir weisen zunächst die erste Behauptung nach. Sei $K$ ein Körper mit Charakteristik $0$.
    Dann definieren wir eine Abbildung $\varphi:\mathbb{Q}\to K$ durch $\varphi(\frac{a}{b})=\frac{\overbrace{1+\ldots+1}^{a}}{\underbrace{1+\ldots+1}_{b}}$,
    mit $a\in \mathbb{Z}$, $b\in\mathbb{N}\setminus\{0\}$. Diese Abbildung ist wohldefiniert,
    da der Nenner laut Voraussetzung niemals $0$ wird und sie unabhängig von der Wahl der Repräsentanten ist
    (kürzbare Ausdrücke in $\mathbb{Q}$ sind auch in $K$ kürzbar).
    Wie man leicht sieht, ist die Abbildung ein Homomorphismus. Und $\varphi$ ist außerdem injektiv, denn gilt
    $\varphi(\frac{a}{b})=0$, so folgt sofort $\frac{a}{b}=0$, da wir $\mathrm{char}(K)=0$ vorausgesetzt haben.
    Da $\varphi(\mathbb{Q})$ einen Unterkörper von $K$ darstellt und $K$ aber ein Primkörper ist, folgt die Surjektivität von $\varphi$,
    also $K\cong \mathbb{Q}$.

    Der Beweis, dass $K\cong\mathbb{Z}_p$ für $\mathrm{char}(K)=p$ gilt, verläuft ähnlich.
    Dieses Mal definieren wir $\varphi:\mathbb{Z}_p\to K$ durch $\varphi(i):=\overbrace{1+\ldots+1}^{i}$.
    Zunächst zeigen wir dieses Mal, dass $\varphi$ ein Homomorphismus ist:
    Für die Addition müssen zwei Fälle unterschieden werden: Falls $i+j<p$ gilt, so folgt klarerweise die
    Verträglichkeit. Ansonsten gilt $i+j=k+p$ mit $0\le k< p$ und wir folgern 
    $\varphi(i+j)=\varphi(k)=\overbrace{1+\ldots+1}^{k}=\overbrace{1+\ldots+1}^{k+p}=\overbrace{1+\ldots+1}^{i}+\overbrace{1+\ldots+1}^{j}=\varphi(i)+\varphi(j)$,
    wobei wir verwendet haben, dass $K$ Charakteristik $p$ hat. Die Verträglichkeit mit der Multiplikation zeigt man ähnlich
    und die Homomorphiebedingung für die neutralen Elemente gilt definitionsgemäß. Die Abbildung
    $\varphi$ ist außerdem injektiv, denn aus $\varphi(i)=0$ folgt klarerweise $i=0$. Damit ist
    $\varphi(\mathbb{Z}_p)$ ein Unterkörper von $K$ und da $K$ ein Primkörper ist, folgt wieder die Surjektivität,
    also $K\cong \mathbb{Z}_p$.


\end{proof}