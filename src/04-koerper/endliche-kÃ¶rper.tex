\section{Endliche Körper}

\begin{theorem}{\ }
    \begin{enumerate}
        \item Sei $K$ ein endlicher Körper mit $\chara K = p \in \mathbb{P}$, so gibt es ein $n \geq 1$ mit $\vert K \vert = p^n$.
        \item Für alle $p \in \mathbb{P}$ und $n \geq 1$ gibt es einen eindeutigen Körper $K$ mit $\chara K = p$ und $\vert K \vert = p^n$.
    \end{enumerate}
\end{theorem}

\begin{proof}{\ }
    \begin{enumerate}
        \item Sei $K \geq \mathbb{Z}_p$ und wähle $n := [K : \mathbb{Z}_p]$, so gilt klarerweise $\vert K \vert = p^n$.
        \item Betrachte
        $$ f(x) = x(x^{p^n - 1} - 1), $$
        so ist
        $$ f'(x) = p^n x^{p^n - 1} - 1 = -1, $$
        also folgt $\ggT(f, f') = 1$, womit die Nullstellen von $f$ nach \cref*{lemma:mehrfache_nullstellen} paarweise verschieden sind. Wähle
        $$ N := \{ \alpha \in Z_{\{f\}}(\mathbb{Z}_p) \mid f(\alpha) = 0 \}, $$
        so gilt gerade $\vert N \vert = p^n = \deg f$. Wir behaupten, dass $N$ ein Körper ist. Klarerweise sind $0, 1 \in N$. Sind $\alpha, \beta \in N$, so ist $\alpha^{p^n} = \alpha, \beta^{p^n} = \beta$, also ist $(\alpha + \beta)^{p^n} = \alpha^{p^n} \in \beta^{p^n} = \alpha + \beta$. Damit ist $\alpha + \beta \in N$. 
        Ist $\alpha\in N$, so gilt $(-\alpha)^{p^n}=(-1)^{p^n}(\alpha)^{p^n}=(-1)^{p^n}\alpha$. Falls $p=2$ ist, so gilt $-1=1$ und daher folgt $-\alpha\in N$. Andernfalls ist $p^n$ ungerade und daher folgt ebenfalls $-\alpha\in N$.
        Entsprechend verifiziert man $\cdot$ und ${}^{-1}$.
    \end{enumerate}
\end{proof}

\begin{remark}
    Für diesen eindeutigen Körper im obigen Satz schreiben wir auch $\mathrm{GF}(p^n)$. Tatächlich gilt der Satz von Wedderburn -- jeder endliche Schiefkörper ist ein Körper, also $\mathrm{GF}(p^n)$ für ein $p\in\mathbb{P}$ und $n\in\mathbb{N}$.
\end{remark}

\begin{lemma}
    Seien $k, n \geq 1, k \mid n$ und $p \in \mathbb{P}$. Dann gilt:
    \begin{enumerate}
        \item $(x^k - 1) \mid (x^n - 1)$
        \item $(p^k - 1) \mid (p^n - 1)$
        \item $(x^{p^k - 1} - 1) \mid (x^{p^n - 1} - 1)$
    \end{enumerate}
\end{lemma}

\begin{proof}{\ }
    \begin{enumerate}
        \item Es gilt $(x^n - 1) = (x^k - 1)(x^{n-k} + x^{n - 2k} + \hdots + x^k + 1)$, da man durch ausmultiplizieren eine Teleskopsumme erhält. 
        \item Folgt aus (1) mit dem Einsetzungshomomorphismus.
        \item Folgt direkt aus (1) und (2).
    \end{enumerate}
\end{proof}

\begin{lemma}
    Seien $K_1, K_2 \leq L, \vert K_1 \vert = \vert K_2 \vert$. Dann gilt sogar $K_1 = K_2$.
\end{lemma}

\begin{proof}
    Wähle $p^n := \vert K_1 \vert = \vert K_2 \vert$ mit $p \in \mathbb{P}, n \geq 1$, so ist $\mathbb{Z}_p \leq K_1, K_2$. Nun ist $K_1$ der Zerfällungskörper von $x^{p^n} - x$, ebenso $K_2$. Nun gilt für alle $\alpha \in K_{1,2}$, dass $\alpha$ eine Nullstelle des besagten Polynoms ist, womit bereits $K_1 = K_2$ folgt.
\end{proof}

\begin{proposition}
    Seien $k, n \geq 1$ und $p \in \mathbb{P}$. Dann existiert ein $K \leq \mathrm{GF}(p^n), \vert K \vert = p^k$ genau dann wenn $k \mid n$.
\end{proposition}

\begin{proof}{\ } \\
    ``$\Rightarrow$'': Es gilt $n = [\mathrm{GF}(p^n) : \mathbb{Z}_p] = [\mathrm{GF}(p^n) : K] \cdot [K : \mathbb{Z}_p] = [\mathrm{GF}(p^n) : K] \cdot k$.

    ``$\Leftarrow$'': Es gilt $g := x^{p^k - 1} - 1 \mid x^{p^n - 1} - 1 =: f$.
    Damit folgt
    $$ \mathbb{Z}_p \leq \mathrm{GF}(p^k) = Z_{\{g\}}(\mathbb{Z}_p) \leq Z_{\{f\}}(\mathbb{Z}_p) = \mathrm{GF}(p^n). $$
\end{proof}

\begin{lemma}
    Sei $n \geq 1, p \in \mathbb{P}$. Dann gilt:
    \begin{enumerate}
        \item Für alle $f \in \mathbb{Z}_p[x]$ irreduzibel, $\deg f = n$, gilt:
        \begin{enumerate}
            \item $\mathrm{GF}(p^n) = Z_{\{f\}}(\mathbb{Z}_p)$
            \item Für alle $\alpha \in \mathrm{GF}(p^n)$ mit $f(\alpha) = 0$ folgt $\mathrm{GF}(p^n) = \mathbb{Z}_p(\alpha)$.
            \item $f \mid x^{p^n} - x$
            \item $f$ hat nur einfache Nullstellen.
        \end{enumerate}
        \item Ist $g \in \mathbb{Z}_p[x]$ irreduzibel, $\deg g = k$, so gilt $g \mid x^{p^n} - x$ genau dann wenn $k \mid n$. Weiters gilt $g^2 \nmid x^{p^n} - x$.
    \end{enumerate}
\end{lemma}

\begin{proof}{\ }
    \begin{enumerate}
        \item \begin{enumerate}
            \item Es gilt $[Z_{\{f\}}(\mathbb{Z}_p) : \mathbb{Z}_p] = n$ und damit $Z_{\{f\}}(\mathbb{Z}_p) = \mathrm{GF}(p^n)$.
            \item Sei $f(\alpha) = 0$. Da $f$ irreduzibel ist folgt $f = \mu_\alpha$. Dann bilden $\alpha^0, \alpha^, \hdots, \alpha^{n-1}$ eine Basis von $\mathrm{GF}(p^n)$ über $\mathbb{Z}_p$, also folgt bereits $\mathbb{Z}_p(\alpha) = \mathrm{GF}(p^n)$.
            \item Sei $\alpha \in \mathrm{GF}(p^n), f(\alpha) = 0$, so gilt $f = \mu_\alpha$ und $\alpha^{p^n} - \alpha = 0$, also gilt $f \mid x^{p^n} - x$.
            \item Aus (c) erhalten wir $f \mid x^{p^n} - x$, wobei $x^{p^n} - x$ nur einfache Nullstellen hat, also hat auch $f$ nur einfache Nullstellen.
        \end{enumerate}
        \item ``$\Rightarrow$'': Es gilt $\mathbb{Z}_p \leq Z_{\{g\}}(\mathbb{Z}_p) \leq \mathrm{GF}(p^n)$. Nach (1a) gilt $Z_{\{g\}}(\mathbb{Z}_p) = \mathrm{GF}(p^k)$, womit $k \mid n$ folgt.
        
        ``$\Leftarrow$'': Nach (1c) gilt $g \mid x^{p^k} - x \mid x^{p^n} - x$. Weiters gilt $g^2 \nmid x^{p^n} - x$, da $x^{p^n} - x$ nur einfache Nullstellen hat.
    \end{enumerate}
\end{proof}

\notedate{15.06.2023}

\begin{definition}
    Ein irreduzibles Polynom $f \in \mathbb{Z}_p[x]$, $\deg f =: n$ heißt \emph{primitiv}\index{Polynom!primitiv}, wenn 
    $$\exists \alpha \in \mathrm{GF}(p^n): f(\alpha) \land \mathrm{GF}(p^n)\setminus\{0\} = \{\alpha^0, \alpha^1, \ldots \}. $$
\end{definition}

\begin{remark}
    In der Tat ist die Forderung $\exists \alpha\in \mathrm{GF}(p^n):f(\alpha)\land \mathrm{GF}(p^n)\setminus\{0\}=\{\alpha^0,\alpha^1,\ldots\}$ in der Definition des primitiven Polynom äquivalent zu $\forall\alpha\in\mathrm{GF}(p^n):(f(\alpha)=0\Rightarrow GF(p^n)\setminus\{0\}=\{\alpha^0,\alpha^1,\ldots\})$. 
\end{remark}

\begin{lemma}
    Sei $K$ ein Körper, $G \le (K \setminus \{0\}, \cdot, 1, {}^{-1})$ eine endliche Untergruppe, dann ist $G$ zyklisch.
\end{lemma}
\begin{proof}
    Es gibt ein $g \in G$, dass für alle $h \in G$ gilt $\ord h \mid \ord g =: \ell \mid \vert G \vert$. Nun ist $h \in G$ eine Nullstelle von $x^\ell-1$ und da dieses nur $\ell$ Nullstellen haben kann, ist $\ell \geq \vert G\vert$. Es gilt daher $\ord g = \vert G \vert$, womit $G$ zyklisch ist.
\end{proof}

\begin{corollary}
    Für $n \in \mathbb{N}$ gibt es ein primitives Polynom in $f \in \mathbb{Z}_p[x]$ vom Grad $n$.
\end{corollary}
\begin{proof}
    Es ist $(\mathrm{GF}(p^n)\setminus\{0\}, \cdot, 1, {}^{-1})$ nach dem vorherigen Lemma zyklisch, womit es ein $\alpha$ gibt, mit $\mathrm{GF}(p^n)\setminus\{0\} = \{\alpha^0, \alpha^1, \ldots\}$. Dann ist $\mu_\alpha$ primitiv.
\end{proof}

\begin{lemma}
    Sei $f \in \mathbb{Z}_p[x]$ irreduzibel, $n := \deg f$, $\alpha_1, \ldots, \alpha_n \in \mathrm{GF}(p^n)$ Nullstellen von $f$. Dann gilt
    \begin{enumerate}
        \item $\forall \varphi \in \Aut(\mathrm{GF}(p^n)): \varphi\vert_{\{\alpha_1, \ldots, \alpha_n\}}$ ist eine Permutation von $\{\alpha_1, \ldots, \alpha_n\}$.
        \item Für alle $i \in \{1, \ldots, n\}$ existiert ein eindeutiger Automorphismus $\varphi_i$ mit $\varphi_i(\alpha_1) = \alpha_i$.
    \end{enumerate}
\end{lemma}
\begin{proof}{\ }
    \begin{enumerate}
        \item Es ist $\varphi \in \Aut(\mathrm{GF}(p^n))$, also gilt $\forall \alpha\in K: f(\alpha) = 0 \Rightarrow f(\varphi(\alpha)) = \varphi(f)(\varphi(\alpha)) = 0$. Da $\varphi\vert_K=\id_K$ gilt und $f$ ein Polynom mit Koeffizienten in $K$ ist, folgt $\varphi(f)=f$ und $\varphi$ bildet Nullstellen von $f$ auf Nullstellen ab.
        Da $\varphi\vert_{\{\alpha_1, \ldots, \alpha_n\}}$ injektiv ist und $\{\alpha_1, \ldots, \alpha_n\}$ endlich ist, ist $\varphi\vert_{\{\alpha_1, \ldots, \alpha_n\}}$ surjektiv und damit eine Permutation.
        \item Es ist $f = \mu_{\alpha_1} = \mu_{\alpha_i}$, also gibt es einen eindeutigen Isomorphismus $\varphi_i: \mathbb{Z}_p(\alpha_1) \to \mathbb{Z}_p(\alpha_i)$ mit $\varphi_i\vert_{\mathbb{Z}_p} = \id_{\mathbb{Z}_p}$ und $\varphi_i(\alpha_1) = \alpha_i$. Da $\mathrm{GF}(p^n) = \mathbb{Z}_p(\alpha_1) = \mathbb{Z}_p(\alpha_i)$ folgt die Aussage.
     \end{enumerate}
\end{proof}

\begin{theorem}{\ }
    \begin{enumerate}
        \item Für alle $k\in\mathbb{N}$ ist die Abbildung $\varphi_k: \mathrm{GF}(p^n) \to \mathrm{GF}(p^n), x \mapsto x^{p^n}$ ein Automorphismus (genannt \emph{Frobniusautomorphismus}). 
        \item $\forall \varphi \in \Aut(\mathrm{GF(p^n)}) \exists k < n: \varphi = \varphi_k$.
    \end{enumerate}
\end{theorem}
\begin{proof}{\ }
    \begin{enumerate}
        \item Es ist wegen \cref*{prop:frob_ringhomom} ein Ringhomomorphismus gegeben. Man sieht leicht, dass $\varphi_k$ sogar ein Körperhomomorphismus ist. Nach \cref*{corollary:koerperhomom_injektiv} ist $\varphi_k$ injektiv und da $\mathrm{GF}(p^n)$ endlich ist, ist die Abbildung surjektiv, also ein Körperautomorphismus.
        
        \item Das vorherige Lemma liefert $\vert \Aut(\mathrm{GF}(p^n)) \vert = n$. Wir müssen nun noch zeigen, dass für $i, j < n$, $i \not= j$ auch $\varphi_i \not= \varphi_j$ gilt. Sei nun $f \in \mathbb{Z}_p[x]$ ein primitives Polynom vom Grad $n$ und $\alpha$ mit $f(\alpha) = 0$ und $\mathrm{GF}(p^n)\setminus\{0\} = \{\alpha^0, \alpha^1, \ldots\}$. Es ist nun $\alpha^{p^i} \not= \alpha^{p^j}$, womit $\varphi_i(\alpha) \not= \varphi_j(\alpha)$ ist.
    \end{enumerate}
\end{proof}

\begin{definition}
    Wir definieren 
        $$ \mathrm{GF}\left(p^\infty\right) := \bigcup_{k \ge 1} \mathrm{GF}\left(p^{k!}\right). $$
\end{definition}
\begin{remark}
    $\mathrm{GF}(p^\infty)$ ist ein Körper, algebraisch über $\mathbb{Z}_p$. Der Beweis davon verläuft analog wie im Beweis der Existenz des algebraischen Abschlusses. Weiter ist $\mathrm{GF}(p^\infty)$ algebraisch abgeschlossen, denn sei $f \in \mathrm{GF}(p^\infty)[x]$ nicht konstant, dann gibt es ein $k$, sodass $f \in \mathrm{GF}(p^{k!})[x]$ und ein $\mathbb{\ell}$, sodass $Z_{\{f\}}(\mathbb{Z}_p) \le \mathrm{GF}(p^{l!})$. Daher hat $f$ eine Nullstelle in $\mathrm{GF}(p^\infty)$.

    Bezeichnet $(\{GF(p^{n})\mid n\in\mathbb{N}\},\iota)$ die Halbordnung aller $GF(p^n)$ mit der Einbettungsabbildung $\iota$, so kann diese Konstruktion auch mit einer beliebigen anderen cofinalen Kette durchgeführt werden. Cofinal bedeutet, dass für jedes $GF(p^n)$ ein Element in der Kette vorkommen muss, welches in der Halbordnung nach $GF(p^n)$ liegt. Offensichtlich erfüllt die von uns gewählte Folge diese Voraussetzungen.
\end{remark}