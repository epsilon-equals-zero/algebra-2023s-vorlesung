\section{Varietäten und Klone}
In diesem Kapitel werden die Begriffe \emph{Varietät} und \emph{Klon} definiert und es werden Beispiel dazu gegeben. Aussagen darüber folgen in den nächsten Kapiteln.

\begin{definition}
    Sei $\Sigma$ eine Menge von Gesetzen über eine Sprache $(f_i)_{i \in I}$, dann heißt die Klasse
    $$ \mathcal{V}(\Sigma) := \left\{ \mathfrak{A} \mid \mathfrak{A} \;\text{ist Algebra über der Sprache}\; (f_i)_{i \in I} \land \forall s \approx t \in \Sigma: A\models s\approx t \right\} $$
    \emph{Varietät}\index{Varietät}. Es handelt sich dabei also um eine durch Gesetze definierte Klasse von Algebren.
\end{definition}

\begin{example}
    Betrachtet man die Sprache $(+, 0, -)$ mit Stelligkeiten $(2, 0, 1)$ und definiert die Gesetzesmenge (mit Variablenmenge $X = \{x,y,z\}$) $\Sigma = \{$
    \begin{itemize}[label={}]
        \item $(x + y) + z \approx x + (y + z)$,
        \item $0 + x \approx x$, $x + 0 \approx x$,
        \item $x + (-x) \approx 0$, $(-x) + x \approx 0$
    \end{itemize}
    $\}$, so ist die Varietät $\mathcal{V}(\Sigma)$ die Klasse aller Gruppen.
    
    Betrachtet man hingegen Gruppen über der Sprache $(+)$ wie in \Cref{rem:alternativegruppe}, so kann man die Gruppenaxiome nicht über Gesetze definieren.
\end{example}

\begin{definition}
    Sei $M$ eine beliebige Menge. Für $1 \le i \le n$ ist die \emph{$n$-dimensionale Projektion auf die $i$-te Komponente}\index{Projektion} definiert als
    $$ \pi_i^{(n)}: M^n \to M, (x_1, \ldots, x_n) \to x_i. $$
\end{definition}

\begin{definition}
    Sei $M$ eine beliebige Menge. Eine Teilmenge von Funktionen $\mathcal{C} \subseteq \bigcup_{n \ge 1} \{f: M^n \to M\}$ heißt \emph{Klon}\index{Klon}, wenn 
    \begin{itemize}
        \item $\mathcal{C}$ alle Projektionen enthält und
        \item $\mathcal{C}$ unter Komposition abgeschlossen ist.
    \end{itemize}

    Die Komposition von $f: M^n \to M$ und $g_1, \ldots, g_n: M^k \to M$ definieren wir hier als 
    $$ f \circ (g_1, \ldots, g_n): M^k \to M, (x_1, \ldots, x_k) \mapsto f(g_1(x_1, \ldots, x_k), \ldots, g_n(x_1, \ldots, x_k)).$$
\end{definition}

\begin{definition}
    Sei $\mathfrak{A} = (A, (f_i)_{i \in I})$ eine Algebra und sei die Menge $\mathcal{T}^{(n)}(\mathfrak{A}) := \{f: A^n \to A\mid f\;\text{ist Termfunktion von}\;\mathfrak{A}\}$. Dann ist $\mathcal{T}(\mathfrak{A}) := \bigcup_{n \ge 1} \mathcal{T}^{(n)}(\mathfrak{A})$ ein Klon und wird \emph{Termklon von $\mathfrak{A}$}\index{Termklon} genannt.
\end{definition}