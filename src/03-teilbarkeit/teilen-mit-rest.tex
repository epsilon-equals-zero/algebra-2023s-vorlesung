\section{Teilen mit Rest}


\begin{example}
    Das folgende Beispiel illustriert die Motivation dieses Kapitels: In den ganzen Zahlen kann die bekannte Division mit Rest, durchgeführt werden. Das heißt für zwei ganze Zahlen $a,b\in \mathbb{Z}$ mit $a\neq 0$ existieren $q,r\in\mathbb{Z}$ sodass $b=qa+r$ gilt, wobei $0\le r<|a|$. Beispielsweise ist $16=5\cdot 3+1$ eine solche Division mit Rest, während $16=4\cdot 3+4$
    diese Definition nicht erfüllt.
\end{example}

\begin{definition}
    Sei $R$ ein Integritätsbereichs-Ring. Dieser heißt \emph{euklidischer Ring}, wenn es eine Funktion $H:R\setminus\{0\}\to\mathbb{N}$ mit
    $$\forall a\in R\setminus\{0\},b\in R:\exists\footnote{Diese Elemente müssen nicht eindeutig sein!} q,r\in R:b=aq+r \quad \land \quad (r=0\lor H(r)<H(a))$$ gibt. Die Funktion $H$ heißt \emph{euklidische Bewertung}.
\end{definition}

\begin{example}
    Ein Beispiel für einen euklidischen Ring ist $\mathbb{Z}$ mit $H(x)=|x|$.
    Weiters ist für einen Körper $K[x]$ ein euklidischer Ring gegeben, wobei die Bewertung der Grad ist. Jeder Körper $K$ mit einer beliebigen Funktion $H:R\setminus\{0\}\to\mathbb{N}$ ist ein triviales Beispiel, da man immer $0$ als Divisionsrest erhalten kann.
\end{example}

\begin{example}
    Wir wir gleich sehen werden, ist jeder euklidische Ring auch ein Hauptidealring. Da
    $\mathbb{Z}[x]$ kein Hauptidealring ist, ist $\mathbb{Z}[x]$ insbesondere kein euklidischer Ring. Ein Beispiel für einen Hauptidealring der kein euklidischer Ring ist, wäre $\mathbb{Z}[\frac{1+\sqrt{-19}}{2}]\subseteq \mathbb{C}$ (ohne Beweis).
\end{example}

\begin{theorem}
    Jeder euklidische Ring ist ein Hauptidealring.
\end{theorem}

\begin{proof}
    Sei $R$ ein euklidischer Ring und $I$ ein Ideal. Falls $I=\{0\}$ ist, so gilt trivialerweise $I=(0)$. Falls $I\neq \{0\}$ gilt, so muss ein $a\in R$ mit $I=(a)=\{aq\mid q\in R\}$ gefunden werden. Wähle daher $a\in I\setminus\{0\}$ mit $H(a)=\min\{H(x)\mid x\in I\}$. Dieses Minimum existiert, da jede nichtleere Teilmenge natürlicher Zahlen ein Minimum hat. Offensichtlich gilt $(a)\subseteq I$.

    Für die andere Mengeninklusion sei $b\in(a)$. Da $R$ ein euklidischer Ring ist, existieren $q,r\in R$ mit $b=aq+r$ und $r=0\lor H(r)<H(a)$. Wegen $r=b-aq\in I$ und der Minimalität von $H(a)$
    folgt, dass $r=0$ gilt, also $b=aq$ und $b\in (a)$.
\end{proof}

\begin{theorem}[Euklidischer Algorithmus]\label{theorem:euklidischer-algorithmus}
    Seien $a,b\in R, a\neq 0$. Wähle $q_1,r_1\in R:b=aq_1+r_1$ mit $r_1=0\lor H(r_1)<H(a)$. Wenn $r_1=0$ ist, dann terminiert der Algorithmus. Ansonsten wählt man $q_2,r_2\in R$ mit
    $a=r_1q_2+r_2$ und $r_2=0\lor H(r_2)<H(r_1)$. Falls $r_2=0$ ist, so terminiert der Algorithmus, ansonsten verfahren wir induktiv. Wenn $r_i, r_{i+1}$ und $q_{i+1}$ bereits gewählt sind, dann wählt man $q_{i+2},r_{i+2}$ mit $r_i=r_{i+1}q_{i+2}+r_{i+2}$ mit $r_{i+2}=0\lor H(r_{i+2})<H(r_{i+1})$. Aufgrund der Schachtelung $H(a)>H(r_1)>H(r_2)$ terminiert der Algorithmus, das heißt es ist $r_k=0$ für ein $k\in\mathbb{N}$. Dann ist $r_{k_1}$ der letzte von $0$ verschieden Rest und es gilt $r_{k-1}=\mathrm{ggT}(a,b)$.
\end{theorem}

\begin{proof}
    Zunächst wird gezeigt, dass $r_{k-1}$ ein Teiler von $a$ und $b$ ist. Das folgt induktiv,
    da $r_{k-1}\mid r_{k-2}$ (wegen $r_k=0$) und $r_{k-1}\mid r_{k-2}q_{k-1}+r_{k-1}=r_{k-3}$.
    Mit Induktion folgt, dass $r_{k-1}\mid a$ und $r_{k-1}\mid b$ gilt.

    Ist nun $t$ ein beliebiger Teiler von $a$ und $b$, so müssen wir zeigen, dass $t\mid r_{k-1}$ gilt.
    Diese Aussage folgt ähnlich da $t\mid b-aq_1=r_1$ und man wieder mit Induktion $t\mid r_{k-1}$ leicht folgert.
    Daher folgt, dass $r_{k-1}=\mathrm{ggT}(a,b)$ gilt. 
\end{proof}

\begin{remark}
    Eine Anwendung des euklidischen Algorithmus ist die Berechnung von Koeffizienten $x,y$ mit
    $ax+by=\mathrm{ggT}(a,b)$. Mit der Notation aus Satz \ref{theorem:euklidischer-algorithmus} folgt
    \begin{align*}
        \mathrm{ggT}(a,b)=&r_{k-1}\\
        &=r_{k-3}-r_{k-2}q_{k-1}\\
        &=r_{k-3}-(r_{k-4}-r_{k-3}q_{k-2})q_{k-1}\\
        &=r_{k-4}(-q_{k-1})+r_{k-3}(1+q_{k-2}q_{k-1})\\
        &=\ldots=ax+by.
    \end{align*}
    Die Koeffizienten $x,y$ sind klarerweise nicht eindeutig, so ist in $\mathbb{Z}$ beispielsweise
    $1=\mathrm{ggT}(5,3)=5(-1)+3\cdot2=5\cdot 2+3(-3).$
\end{remark}

\begin{remark}
    Wir wollen an dieser Stelle noch einen kurzen Überblick über die verschiedenen Arten von Ringen
    geben und vor allem auch auf die Unterschiede über darin gültigen Aussagen eingehen.

    In Faktoriellen Ringen gibt es zu $a,b\in R$ einen größten gemeinsamen Teiler $\mathrm{ggT}(a,b)$.

    In einem Hauptidealring gibt es nicht nur den größten gemeinsamen Teiler, sondern dieser kann auch als Linearkombination
    dargestellt werden, das heißt für $a,b\in R$ existieren $x,y\in R$ mit $\mathrm{ggT}(a,b)=ax+by$.

    In einem euklidischen Ring gibt es den ggT, dieser kann linearkombiniert werden und mithilfe des euklidischen Algorithmus
    können Faktoren berechnet werden.
\end{remark}

\begin{proposition}
    Sei $R$ ein faktorieller Ring, $K$ der Quotientenkörper und $\frac{p}{q}\in K$. Dann gibt es
    $p',q'\in R, q'\neq 0$ sodass $\frac{p}{q}=\frac{p'}{q'}$ und $\mathrm{ggT}(p',q')=1$. Wenn
    $\frac{p''}{q''}=\frac{p}{q}$ mit $\mathrm{ggT}(p'',q'')=1$, dann gilt $p''\sim p'$ und $q''\sim q'$.
\end{proposition}

\begin{proof}
    Mit $p':=\frac{p}{\mathrm{ggT}(p,q)}$ und $q':=\frac{q}{\mathrm{ggT}(p,q)}$ folgt die Existenz,
    wobei man $\frac{p'}{q'}=\frac{p}{q}$ über die Primfaktorenzerlegung nachweist.
    Ist $\frac{p''}{q''}$ ebenfalls eine solche Darstellung von $\frac{p}{q}$, so überzeugt man sich
    von $p'\sim p''$ und $q'\sim q''$ ebenfalls mithilfe der Primfaktorenzerlegung in $R$.
\end{proof}