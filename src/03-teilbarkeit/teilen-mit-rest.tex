\section{Teilen mit Rest}


\begin{example}
    Das folgende Beispiel illustriert die Motivation dieses Kapitels: In den ganzen Zahlen kann die bekannte Division mit Rest, durchgeführt werden. Das heißt für zwei ganze Zahlen $a,b\in \mathbb{Z}$ mit $a\neq 0$ existieren $q,r\in\mathbb{Z}$ sodass $b=qa+r$ gilt, wobei $0\le r<|a|$. Beispielsweise ist $16=5\cdot 3+1$ eine solche Division mit Rest, während $16=4\cdot 3+4$
    diese Definition nicht erfüllt.
\end{example}

\begin{definition}
    Sei $R$ ein Ring und Integritätsbereich. Dieser heißt \emph{euklidischer Ring}\index{Ring!euklidisch}, wenn es eine Funktion $H:R\setminus\{0\}\to\mathbb{N}$ mit
    $$\forall a\in R\setminus\{0\},b\in R\exists\footnote{Diese Elemente müssen nicht eindeutig sein!} q,r\in R:b=aq+r \quad \land \quad (r=0\lor H(r)<H(a))$$ gibt. Die Funktion $H$ heißt \emph{euklidische Bewertung}\index{euklidische Bewertung}.
\end{definition}

\begin{example}
    Ein Beispiel für einen euklidischen Ring ist $\mathbb{Z}$ mit $H(x)=|x|$.
    Weiters ist für einen Körper $K$ der Polynomring $K[x]$ ein euklidischer Ring, wobei die Bewertung der Grad ist. Jeder Körper $K$ mit einer beliebigen Funktion $H:R\setminus\{0\}\to\mathbb{N}$ ist ein triviales Beispiel, da man immer $0$ als Divisionsrest erhalten kann.
\end{example}

\begin{example}
    Wie wir gleich sehen werden, ist jeder euklidische Ring auch ein Hauptidealring. Da
    $\mathbb{Z}[x]$ kein Hauptidealring ist, ist $\mathbb{Z}[x]$ insbesondere kein euklidischer Ring. Ein Beispiel für einen Hauptidealring der kein euklidischer Ring ist, wäre $\mathbb{Z}[\frac{1+\sqrt{-19}}{2}]\subseteq \mathbb{C}$ (ohne Beweis).
\end{example}

\begin{theorem}
    Jeder euklidische Ring ist ein Hauptidealring.
\end{theorem}
\begin{proof}
    Sei $R$ ein euklidischer Ring und $I$ ein Ideal. Falls $I=\{0\}$ ist, so gilt trivialerweise $I=(0)$. Falls $I\neq \{0\}$ gilt, so muss ein $a\in R$ mit $I=(a)=\{aq\mid q\in R\}$ gefunden werden. Wähle daher $a\in I\setminus\{0\}$ mit $H(a)=\min\{H(x)\mid x\in I\}$. Dieses Minimum existiert, da jede nichtleere Teilmenge natürlicher Zahlen ein Minimum hat. Offensichtlich gilt $(a)\subseteq I$.

    Für die andere Mengeninklusion sei $b\in(a)$. Da $R$ ein euklidischer Ring ist, existieren $q,r\in R$ mit $b=aq+r$ und $r=0\lor H(r)<H(a)$. Wegen $r=b-aq\in I$ und der Minimalität von $H(a)$
    folgt, dass $r=0$ gilt, also $b=aq$ und $b\in (a)$.
\end{proof}

\begin{theorem}[Euklidischer Algorithmus]\label{theorem:euklidischer-algorithmus}\index{euklidischer Algorithmus}
    Seien $a,b\in R, a\neq 0$. Wähle $q_1,r_1\in R:b=aq_1+r_1$ mit $r_1=0\lor H(r_1)<H(a)$. Wenn $r_1=0$ ist, dann terminiert der Algorithmus. Ansonsten wählt man $q_2,r_2\in R$ mit
    $a=r_1q_2+r_2$ und $r_2=0\lor H(r_2)<H(r_1)$. Falls $r_2=0$ ist, so terminiert der Algorithmus, ansonsten verfahren wir induktiv. Wenn $r_i, r_{i+1}$ und $q_{i+1}$ bereits gewählt sind, dann wählt man $q_{i+2},r_{i+2}$ mit $r_i=r_{i+1}q_{i+2}+r_{i+2}$ mit $r_{i+2}=0\lor H(r_{i+2})<H(r_{i+1})$. Aufgrund der Schachtelung $H(a)>H(r_1)>H(r_2)$ terminiert der Algorithmus, das heißt es ist $r_k=0$ für ein $k\in\mathbb{N}$. Dann ist $r_{k_1}$ der letzte von $0$ verschieden Rest und es gilt $r_{k-1}=\mathrm{ggT}(a,b)$.
\end{theorem}

\begin{proof}
    Zunächst wird gezeigt, dass $r_{k-1}$ ein Teiler von $a$ und $b$ ist. Das folgt induktiv,
    da $r_{k-1}\mid r_{k-2}$ (wegen $r_k=0$) und $r_{k-1}\mid r_{k-2}q_{k-1}+r_{k-1}=r_{k-3}$.
    Mit Induktion folgt, dass $r_{k-1}\mid a$ und $r_{k-1}\mid b$ gilt.

    Ist nun $t$ ein beliebiger Teiler von $a$ und $b$, so müssen wir zeigen, dass $t\mid r_{k-1}$ gilt.
    Diese Aussage folgt ähnlich da $t\mid b-aq_1=r_1$ und man wieder mit Induktion $t\mid r_{k-1}$ leicht folgert.
    Daher folgt, dass $r_{k-1}=\mathrm{ggT}(a,b)$ gilt. 
\end{proof}

\begin{remark}
    Eine Anwendung des euklidischen Algorithmus ist die Berechnung von Koeffizienten $x,y$ mit
    $ax+by=\mathrm{ggT}(a,b)$. Mit der Notation aus Satz \ref{theorem:euklidischer-algorithmus} folgt
    \begin{align*}
        \mathrm{ggT}(a,b)&=r_{k-1}\\
        &=r_{k-3}-r_{k-2}q_{k-1}\\
        &=r_{k-3}-(r_{k-4}-r_{k-3}q_{k-2})q_{k-1}\\
        &=r_{k-4}(-q_{k-1})+r_{k-3}(1+q_{k-2}q_{k-1})\\
        &=\ldots=ax+by.
    \end{align*}
    Die Koeffizienten $x,y$ sind klarerweise nicht eindeutig, so ist in $\mathbb{Z}$ beispielsweise
    $1=\mathrm{ggT}(5,3)=5(-1)+3\cdot2=5\cdot 2+3(-3).$
\end{remark}

\begin{remark}
    Wir wollen an dieser Stelle noch einen kurzen Überblick über die verschiedenen Arten von Ringen
    geben und vor allem auch auf die Unterschiede über darin gültigen Aussagen eingehen.

    In Faktoriellen Ringen gibt es zu $a,b\in R$ einen größten gemeinsamen Teiler $\mathrm{ggT}(a,b)$.

    In einem Hauptidealring gibt es nicht nur den größten gemeinsamen Teiler, sondern dieser kann auch als Linearkombination
    dargestellt werden, das heißt für $a,b\in R$ existieren $x,y\in R$ mit $\mathrm{ggT}(a,b)=ax+by$.

    In einem euklidischen Ring gibt es den ggT, dieser kann linearkombiniert werden und mithilfe des euklidischen Algorithmus
    können Faktoren berechnet werden.
\end{remark}

\begin{proposition}
    Sei $R$ ein faktorieller Ring, $K$ der Quotientenkörper und $\frac{p}{q}\in K$. Dann gibt es
    $p',q'\in R, q'\neq 0$ sodass $\frac{p}{q}=\frac{p'}{q'}$ und $\mathrm{ggT}(p',q')=1$. Wenn
    $\frac{p''}{q''}=\frac{p}{q}$ mit $\mathrm{ggT}(p'',q'')=1$, dann gilt $p''\sim p'$ und $q''\sim q'$.
\end{proposition}

\begin{proof}
    Mit $p':=\frac{p}{\mathrm{ggT}(p,q)}$ und $q':=\frac{q}{\mathrm{ggT}(p,q)}$ folgt die Existenz,
    wobei man $\frac{p'}{q'}=\frac{p}{q}$ über die Primfaktorenzerlegung nachweist.
    Ist $\frac{p''}{q''}$ ebenfalls eine solche Darstellung von $\frac{p}{q}$, so überzeugt man sich
    von $p'\sim p''$ und $q'\sim q''$ ebenfalls mithilfe der Primfaktorenzerlegung in $R$.
\end{proof}

\notedate{24.05.2023}

\section{Der Satz von Gauß}

\begin{theorem}[Satz von Gauß]
    Ist $R$ ein faktorieller Ring, so ist auch $R[x]$ faktoriell.
\end{theorem}

\begin{corollary}
    Sei $R$ ein faktorieller Ring. Dann gilt:
    \begin{itemize}
        \item Der Polynomring $R[x_1, \hdots, x_n]$ ist faktoriell.
        \item Ist $X$ eine beliebige Menge, so ist auch $R[X]$ faktoriell.
    \end{itemize}
\end{corollary}

\begin{corollary}
    $\mathbb{Z}[x]$ ist faktoriell.
\end{corollary}

\begin{definition}
    Ist $R$ ein Ring und $f = \sum_{i=0}^n a_i x^i \in R[x]$, so nennen wir $f$ \emph{leer}\index{Polynom!leer} (oder auch \emph{primitiv}\index{Polynom!primitiv}), wenn
    $$ \mathrm{ggT}( a_0, \hdots, a_n ) = 1. $$
\end{definition}

\begin{remark}
    Ist $R$ ein faktorieller Ring, so existiert für alle $f \in R[x]$ eine Darstellung
    $$ f = \mathrm{ggT}(a_0, \hdots, a_n) \cdot f_0, $$
    wobei $f_0 \in R[x]$ leer ist.
\end{remark}

\begin{lemma}
    Sei $R$ faktoriell, $f, g \in R[x]$, $p \in R$ prim. Dann gilt
    $$ p \mid f g \Rightarrow p \mid f \lor p \mid g. $$
\end{lemma}

\begin{proof}
    Wir zeigen die Aussage mittels Induktion nach $\deg fg = n+m$, wobei
    $$ f = \sum_{i=0}^n a_i x^i,\quad g = \sum_{j=0}^n b_j x^j. $$

    Induktionsanfang ($n+m=0$): Es sind $f, g \in R$, womit aus $p \mid fg$ folgt $p \mid f \lor p \mid g$, da $p$ prim in $R$ ist.

    Induktionsschritt ($n+m \to n+m+1$): Gilt $p \mid fg$, so gilt $p \mid a_n b_m$, da $a_n b_m$ der Leitkoeffizient ist und damit, da $p$ prim in $R$ ist, $p \mid a_n \lor p \mid b_m$. Nehmen wir \obda $p \mid a_n$ an. Schreiben wir nun $f = a_n x^n + f'$. Es gilt
    $$ fg = a_n x^n g + f' g. $$
    Nun teilt $p$ jedoch $fg, a_n$ und damit auch $f' g$. Nach Induktionsvoraussetzung gilt damit $p \mid f' \lor p \mid g$, und damit entweder direkt die Beauptung oder $p \mid a_n, p \mid f'$ und damit $p \mid f$.
\end{proof}

\begin{corollary}
    Sei $R$ faktoriell, $f,g \in R[x]$ leer, so ist auch $fg$ leer.
\end{corollary}

\begin{lemma}
    Sei $R$ faktoriell, $Q$ der Quotientenkörper von $R$ und $f \in Q[x]$. Dann existieren $c_f \in Q, f_0 \in R[x]$ leer, mit
    $$ f = c_f \cdot f_0. $$

    Diese Darstellung ist eindeutig bis Multiplikation mit einer Einheit (aus $R$).

    Weiters gibt es zu $f, g \in Q[x]$ eine Einheit $e \in R$ mit
    $$ c_{f \cdot g} = e \cdot c_f \cdot c_g. $$
\end{lemma}

\begin{proof}
    Die Koeffizienten von $f$ in $Q$ haben eine Darstellung als Quotient mit teilerfremden Elementen aus dem Ring, wir können also schreiben
    $$ f = \sum_{i=0}^\ell a_i x^i = \sum_{i=0}^\ell \frac{z_i}{n_i} x^i = \frac{\mathrm{ggT}(z_0, \hdots, z_n)}{\mathrm{kgV}(n_0, \hdots, n_\ell)} \sum_{i=0}^\ell b_i x^i, $$
    wobei $b_i \in R$ teilerfremd und sich somit sofort die geforderte Darstellung ergibt.

    Seien nun $c_f \cdot f_0 = f = d \cdot g$ zwei Darstellungen. Schreiben wir
    $$ f_0 = \sum_{i=0}^\ell b_i x^i,\quad g = \sum_{i=0}^\ell t_i x^i, $$
    so folgt durch Koeffizientenvergleich, dass für alle $i$ gilt $c_f \cdot b_i = d \cdot t_i$. Schreiben wir $c_f = \frac{c_f^z}{c_f^n}, d = \frac{d^z}{d^n}$. Damit gilt $c_f^z b_i d^n = d^z t_i c_f^n$, aufgrund der Eindeutigkeit des ggT's (bis auf Assoziiertheit) folgt $c_f^z d^n \sim d^z c_f^n$ und damit die Existenz einer Einheit $e$ mit $e \cdot c_f = d$.

    Sind nun $f = c_f \cdot f_0, g = c_g \cdot g_0$, so folgt
    $$ f \cdot g = (c_f \cdot c_g) \cdot (f_0 \cdot g_0) $$
    und damit sofort die Aussage.
\end{proof}

\begin{lemma}
    Sei $R$ faktoriell und $Q$ der Quotientenkörper von $R$. Sei $f \in R[x]$ irreduzibel in $R[x]$, $\deg f \geq 1$, so ist $f$ irreduzibel in $Q[x]$.
\end{lemma}

\begin{proof}
    Sei $f = g \cdot h$, $g, h \in Q[x]$. Gilt $\deg g = 0 \lor \deg h = 0$, so folgt sofort die Assoziiertheit von $g$ oder $h$ zu $1$ in $Q[x]$. Sind $\deg f, \deg h \geq 1$, so schreibe mit obigem Lemma $g = c_g \cdot g_0, h = c_h \cdot h_0$. Wir nehmen \obda $c_f = c_g \cdot c_h$ an. $f$ ist irreduzibel in $R[x]$, insbesondere ist $f$ also leer. Wir können also \obda $c_f = 1$ annehmen. Damit ist also
    $$ f = g \cdot h = c_g \cdot g_0 \cdot c_h \cdot h_0 = g_0 \cdot h_0, $$
    im Widerspruch dazu, dass $f$ irreduzibel in $R[x]$ ist.
\end{proof}

\begin{lemma}
    Sei $R$ ein faktorieller Ring. Ist $f \in R[x]$ irreduzibel, so ist $f$ prim.
\end{lemma}

\begin{proof}
    Seien $g, h \in R[x], f \mid g \cdot h$. Wir wollen $f \mid g \lor f \mid h$ zeigen. In $Q[x]$ gilt $f \mid g \lor f \mid h$, \obda sei $p \in Q[x]$ mit $f \cdot p = g$. Nun können wir also
    $$ f \cdot c_p \cdot p_0 = g = c_g \cdot g_0 $$
    schreiben, also $c_p \cdot (f \cdot p_0) = c_g \cdot g_0$. Aufgrund der Eindeutigkeit dieser Darstellung gibt es eine Einheit $e \in R$ mit $c_p = e \cdot c_g$. Damit ist jedoch auch $c_p \in R$, womit $p \in R[x]$ folgt. Damit gilt $f \mid g$ in $R[x]$.
\end{proof}

\begin{proof}[Beweis (Satz von Gauß)]
    Sei $f \in R[x]$, dann ist
    $$ f = c_f \cdot f_0 = c_f^1 \cdot \hdots \cdot c_f^n \cdot f_0^1 \cdot \hdots \cdot f_0^\ell, $$
    wobei die erste Zerlegung in Primelemente existiert da $R$ faktoriell ist und $c_f^i$ prim in $R[x]$ ist, nach obigem Lemma. Letztere Zerlegung in irreduzible Polynome existiert aus Gradgründen, wobei $f_0^j$ prim nach obigem Lemma sind.
\end{proof}