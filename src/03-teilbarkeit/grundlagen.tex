\section{Grundlagen}

\begin{definition}
    Sei $(H,\cdot)$ eine Halbgruppe und $a,b\in H$. Dann sind definiert:
    \begin{itemize}
        \item $a\mid b:\Leftrightarrow \exists c\in H:a\cdot c=b$ \tab (\emph{$a$ teilt $b$}\index{teilt})
        \item $a\sim b:\Leftrightarrow a\mid b\land b\mid a$ \tab (\emph{$a$ ist assoziiert zu $b$}\index{ist assoziiert zu})
    \end{itemize}
\end{definition}

\begin{remark}
    Ist $(H,\cdot)$ eine Halbgruppe, so ist die Teilbarkeitsrelation $\mid$ transitiv.
    Falls $H$ ein neutrales Element $e$ besitzt, so ist $\mid$ auch reflexiv. Relationen mit diesen beiden Eigenschaften
    werden auch \emph{Quasiordnung} genannt. Im Falle eines kommutativen Monoides handelt es sich bei
    $\sim$ um eine Kongruenzrelation.
\end{remark}

\begin{example}
    In $(\mathbb{Z},\cdot)$ gilt beispielsweise für alle $a\in\mathbb{Z}:a\mid a$ und $a\mid -a$. 
\end{example}

\begin{proposition}
    Sei $R$ ein kommutativer Ring mit $1$ und $p\in R$. Dann sind äquivalent:
    \begin{enumerate}
        \item $(p)\vartriangleleft R$ ist prim.
        \item $p\not\sim 1$ und für alle $a,b\in R$ folgt
        aus $p\mid a\cdot b$, dass $p\mid a$ oder $p\mid b$ gilt.
    \end{enumerate}
\end{proposition}

\begin{proof}{\ }
    \begin{itemize}[leftmargin=2.5cm] 
        \item[$(1) \Rightarrow (2)$:]Da $(p)$ prim ist, ist das erzeugte Ideal insbesondere echt, daher ist $1\not\in (p)$, also gilt $p \nmid 1$ und $p\not\sim 1$.
        Seien $a,b\in R$ beliebig mit $p\mid a\cdot b$. Dann ist $ab\in (p)$, also $a\in(p)$ oder $b\in(p)$, da $(p)$ prim ist.
        Das ist aber äquivalent zu $p\mid a$ oder $p\mid b$.
        \item[$(1) \Leftarrow (2)$:]Da $p\not\sim 1$ gilt, folgt dass $(p)\neq R$ ist, also ist das erzeugte Ideal echt.
        Seien weiters $a,b\in R$ mit $a,b\in (p)$. Dann gilt $p\mid ab$ und gemäß Voraussetzung folgt $p\mid a$ oder $p\mid b$.
        Das ist widerum äquivalent zu $a\in(p)$ oder $b\in(p)$.
    \end{itemize}
\end{proof}

\begin{definition}
    Sei $R$ ein kommutativer Ring mit $1$ und $p\in R$. Dann heißt $p$
    \begin{itemize}
        \item \emph{prim}\index{prim} $:\Leftrightarrow p\neq 0, p\not\sim 1\land \forall a,b\in R:p\mid ab\Rightarrow p\mid a\lor p\mid b,$
        \item \emph{irreduzibel}\index{irreduzibel} $:\Leftrightarrow p\not\sim 1\land \forall a,b\in R:ab=p\Rightarrow a\sim 1\lor b\sim 1.$
    \end{itemize}
\end{definition}

\begin{proposition}
    Sei $R$ ein Integritätsbereich und $p\in R$. Dann folgt wenn $p$ prim ist, dass $p$ auch irreduzibel ist.
\end{proposition}

\begin{proof}
    Seien $a,b\in R$ mit $ab=p$. Dann gilt nach Definition $p\mid ab$, also $p\mid a$ oder $p\mid b$.
    \obda gelte $p\mid a$, das heißt es existiert $c\in R$ sodass $pc=a$. Dann gilt $p=pcb\Leftrightarrow p(1-cb)=0$
    und da $p\neq 0$ ist und $R$ ein Integritätsbereich ist, folgt $1-cb=0$, also $cb=1$ und $b\sim 1$.
\end{proof}

\begin{example}
    Die Umkehrung dieser Proposition stimmt nicht.
    Durch $\mathbb{Z}[\sqrt{-5}] := \{a+b\sqrt{-5} \mid a,b \in \mathbb{Z}\}$ ist ein Integritätsbereich gegeben, in welchem es irreduzible Element
    gibt, welche nicht prim sind, beispielsweise $2$ oder $3$ (siehe Übung).
\end{example}