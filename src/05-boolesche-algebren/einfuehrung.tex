\section{Einführung}

Zuerst sei an die \cref{def:boolsche_algebra} einer \emph{Booleschen Algebra} erinnert.

\begin{example}
    Für eine Menge $M$ ist $(\mathcal{P}(M), \cap, \cup, \emptyset, M, \overline{\cdot})$ eine boolsche Algebra. $\{A \subseteq M \mid \vert A\vert < \infty \lor \vert \overline{A} \vert < \infty\}$ ist eine Unteralgebra davon.
\end{example}
\begin{example}
    $\mathfrak{B}_2 = ({0,1}, \land, \lor, 0, 1, \lnot)$ ist eine boolsche Algebra.
\end{example}

\begin{lemma}
    Sei $\mathfrak{B}$ eine boolesche Algebra und $a,b \in B$.
    \begin{enumerate}
        \item Gilt $a \lor b = 1, a \land b = 0$, dann gilt $a' = b$.
        \item Es ist $a'' = a$.
        \item $0' = 1, 1' = 0$.
        \item $(a \lor b)' = a' \land b'$, $(a \land b)' = a' \lor b'$.
    \end{enumerate}
\end{lemma}
\begin{proof}{\ }
    Wir zeigen 1., der Rest folgt daraus.
    % TODO
\end{proof}