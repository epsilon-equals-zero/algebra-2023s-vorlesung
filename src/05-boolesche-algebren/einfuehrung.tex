\section{Einführung}

Zuerst sei an die \cref{def:boolsche_algebra} einer \emph{Booleschen Algebra} erinnert.

\begin{example}
    Für eine Menge $M$ ist $(\mathcal{P}(M), \cap, \cup, \emptyset, M, \overline{\cdot})$ eine boolsche Algebra. $\{A \subseteq M \mid \vert A\vert < \infty \lor \vert \overline{A} \vert < \infty\}$ ist eine Unteralgebra davon. Im Fall, dass $|M|=|\mathbb{N}|$ gilt, liefert das zweite Beispiel eine abzählbare boolsche Algebra, während die $\mathcal{P}(M)$ dann überabzählbar ist.
\end{example}
\begin{example}
    $\mathfrak{B}_2 = ({0,1}, \land, \lor, 0, 1, \lnot)$ ist eine boolsche Algebra. Diese ist offensichtlich die einzige zweielementige Boolesche Algebra und erzeugt die gesamte Varietät der boolschen Algebren. Mithilfe von Produktbildung erhält man eine zu $(\mathcal{P},\cap,\cup,\emptyset,M,\overline{\cdot})$ isomorphe Algebra. Nach dem nochfolgenden Darstellungssatz ist jede boolesche Algebra isomorph zu einer Unteralgebra einer Boolschen Algebra von diesem Typ. Das heißt wenn $\mathcal{K}$ die Varietät der Booleschen Algebren bezeichnet, so gilt $\mathcal{K}=SP(\mathfrak{B}_2)$. Insbesondere gelten in $\mathfrak{B}_2$ genau alle Gesetze die in der gesamten Varietät gelten.
\end{example}

\begin{example}
    Für einen beliebigen topologischen Raum bilden die clopen Mengen (also jene die sowohl offen als auch abgeschlossen sind) eine Boolesche Algebra, analog wie beim ersten Beispiel.
\end{example}

\begin{example}
    Die freie Boolesche Algebra über der Variablenmenge $\{x_1,x_2,\ldots\}$ bildet ebenfalls eine interessante Boolesche Algebra. Sie ist bis auf Isomorphie die einzige abzählbare Boolesche Algebra ohne Atome, das heißt für jeden Term $t\neq 0$ (wobei die Terme nach den Gesetzen der Varietät faktorisiert werden) existiert ein Element $s\neq 0$ das in der induzierten Halbordnung unter $t$ liegt. Insbesondere existieren auch keine Co-Atome.
\end{example}

\begin{lemma}
    Sei $\mathfrak{B}$ eine boolesche Algebra und $a,b \in B$.
    \begin{enumerate}
        \item Gilt $a \lor b = 1, a \land b = 0$, dann gilt $a' = b$ (das heißt die Abbildung $'$ ist eindeutig festgelegt).
        \item Es ist $a'' = a$.
        \item $0' = 1, 1' = 0$.
        \item $(a \lor b)' = a' \land b'$, $(a \land b)' = a' \lor b'$.
    \end{enumerate}
\end{lemma}
\begin{proof}{\ }
    Wir zeigen zunächst den ersten Punkt. Es gilt $\overbrace{(a\lor b)}^{=1}\land a'=a'$. Nach dem Distributivgesetz gilt aber auch $(a\lor b)\land a'=(a\land a')\lor (b\land a')=b\land a'$. Insgesamt folgt also $a'=b\land a'$ und daher in der induzierten Halbordnung gilt $a'\le b$. Analog zeigt man $b\leq a'$ und es folgt $a'=b$.

    Wendet man das gezeigt auf $a'$ und $a$ an, so erhält man $a=a''$. Offensichtlich aus dem ersten Punkt auch sofort $0'=1$ und $1'=0$.

    Wegen $(a\lor b)\lor (a'\land b')=((a\lor b)\lor a') \land ((a\lor b)\lor b')=1\land 1=1$ und $(a\lor b)\land (a'\land b')=(a\land (a'\land b'))\lor (b\land(a'\land b))=0\lor 0=0$ folgt wieder aus dem ersten Punkt $(a\lor b)'=a'\land b'$. Das zweite Gesetz folgt aus diesem gemeinsam mit dem zweiten Punkt.
\end{proof}

\notedate{21.06.2023}

\begin{definition}
    Sei $R$ ein kommutativer Ring mit 1, dann heißt $R$ \emph{Boolesch}, wenn
    $$ \forall x \in R : x \cdot x = x. $$
\end{definition}

\begin{lemma}
    Sei $R$ ein Boolescher Ring, dann gilt
    $$ \forall x \in R : x = -x. $$
\end{lemma}

\begin{proof}
    Es gilt
    $$ 1 + x = (1+x)(1+x) = x \cdot x + x + x + 1 = x + x + x + 1, $$
    womit durch Kürzen $ 0 = x + x $ folgt.
\end{proof}

\begin{proposition}{\ }
    \begin{enumerate}
        \item Sei $R$ ein Boolescher Ring. Dann definieren wir die Operationen\footnote{Mit dem obigen Lemma sehen wir $x \lor y = x + y - x \cdot y$.}
        $$ x \land y := x \cdot y, \quad x \lor y := x + y + x \cdot y, \quad x' := -x. $$
        Dann ist $(R, \land, \lor, 0, 1, ')$ eine Boolesche Algebra.

        \item Sei $B$ eine Boolesche Algebra. Dann definieren wir die Operationen
        $$ x \cdot y := x \land y, \quad x + y := (x \land y') \lor (y \land x') =: x \Delta y, \quad -x := x. $$
        Dann ist $(B, +, 0, -, \cdot, 1)$ ein Boolescher Ring.

        \item Die durch die obigen Operationen beschriebenen Übersetzungen von Booleschen Ringen auf Boolesche Algebren und umgekehrt sind zueinander invers.

        \item Die Homomorphismen auf den Strukturen übersetzen sich (muss ma sich überlegen wie ma des aufschreibt)
    \end{enumerate}
\end{proposition}

\begin{proof}{\ }
    \begin{enumerate}
        \item Folgt durch Nachrechnen.
        \item Folgt durch Nachrechnen.
        \item Folgt durch Nachrechnen.
        \item Es gilt $T(R_1) = T(B_1)$. Da die Eigenschaft ein Homomorphismus zu sein lediglich von den Termfunktionen abhängt, und die Termfunktionen übereinstimmen, ist auch der \\ Homomorphismus-Begriff derselbe.
    \end{enumerate}
\end{proof}

\begin{definition}
    Sei $\mathfrak{V}$ ein Verband und sei $\emptyset \neq A \subseteq \mathfrak{V}$. Dann heißt $A$ \emph{Filter}, wenn
    \begin{itemize}
        \item $\forall x \in A \forall y \in \mathfrak{V} : ( x \leq y \implies y \in A )$,
        \item $\forall x, y \in A : x \land y \in A$.
    \end{itemize}
    Ist $A$ ein Filter, so nennen wir $A$ \emph{prim}, wenn
    $$ \forall x ,y \in \mathfrak{V} : ( x \lor y \in A \implies x \in A \lor y \in A ). $$
    
\end{definition}

\begin{definition}
    Sei $\mathfrak{V}$ ein Verband und sei $\emptyset \neq A \subseteq \mathfrak{V}$. Dann heißt $A$ \emph{Ideal}, wenn
    \begin{itemize}
        \item $\forall x \in A \forall y \in \mathfrak{V} : ( y \leq x \implies y \in A )$,
        \item $\forall x, y \in A : x \lor y \in A$.
    \end{itemize}
    Ist $A$ ein Ideal, so nennen wir $A$ \emph{prim}, wenn
    $$ \forall x, y \in \mathfrak{V} : ( x \land y \in A \implies x \in A \lor y \in A ). $$
\end{definition}

\begin{proposition}
    Seien $R$ und $B$ die zueinander ``assoziierten'' Booleschen Ringe, beziehungsweise Boolesche Algebren. Ist $I \subseteq B$ ein Ideal von $B$, so ist $I$ ein Ideal von $R$. Ist entsprechend $I \subseteq R$ ein Ideal von $R$, so ist $I$ ein Ideal von $B$.
\end{proposition}

\begin{proof}{\ }
    \begin{enumerate}
        \item Seien $x, y \in I$. Dann ist $x + y = (x \land y') \lor (y \land x') \in I$.

        Sind $x \in I, y \in R$, so ist $x \cdot y = x \land y \in I$.

        \item Seien $x, y \in I$. Dann ist $x \lor y = x + y + x \cdot y \in I$.
        
        Sind $x \in I, y \leq x$, so ist $y = y \land x = y \cdot x \in I$.
    \end{enumerate}
\end{proof}

\begin{remark}
    Sei $B$ eine Boolesche Algebra und $F \subseteq B$ ein Filter. Dann ist
    $$ \{ a' \mid a \in F \} $$
    ein Ideal.
\end{remark}

\begin{theorem}[Homomorphiesatz für Boolesche Algebren]
    Seien $B_1, B_2$ Boolesche Algebren und $f : B_1 \to B_2$ ein surjektiver Homomorphismus. Dann induziert dies einen Isomorphismus $ \widetilde{f} : B_1 /_{\ker f} \to B_2 $.
\end{theorem}

\begin{example}
    Betrachte $\mathcal{P}(\mathbb{N})$ und $I := \{ A \in \mathcal{P}(\mathbb{N}) \mid A \text{ endlich} \}$, so ist $I$ ein Ideal. Da $I$ abzählbar ist, ist jedoch $\mathcal{P}(\mathbb{N}) /_I$ immer noch ``groß''.
\end{example}

\begin{definition}
    Sei $B$ eine Boolesche Algebra. Dann heißt $a \in B$ \emph{Atom}, wenn $a \neq 0$ und $\forall b \in B, b \neq 0, b \neq a: b \nleq a$. $a$ ist also ein minimales Element, wenn wir die 0 nicht berücksichtigen.
\end{definition}