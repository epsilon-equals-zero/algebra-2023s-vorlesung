\section{Der Satz von Stone}

\begin{theorem}[Satz von Stone]{\ }
    \begin{itemize}
        \item Sei $B$ eine Boolesche Algebra. dann gibt es eine Menge $M$ und einen injektiven Homomorphismus $f : B \to \mathcal{P}(M)$.
        \item Sei $B$ eine endliche Boolesche Algebra. Dann gibt es eine Menge $M$ und einen bijektiven Homomorphismus $f : B \to \mathcal{P}(M)$.
    \end{itemize}
\end{theorem}

\begin{corollary}
    Ist $B$ eine Boolesche Algebra, so ist $B \in \mathrm{SP}(\mathfrak{B}_2)$.
\end{corollary}

\begin{proof}
    todo
\end{proof}

\begin{lemma}
    Sei $B$ eine Boolesche Algebra, $\emptyset \neq I \subsetneq B$ ein Ideal. Dann ist $I$ ein maximales echtes Ideal genau dann wenn
    $$ \forall a \in B : (a \in I \lor a' \in I). $$
\end{lemma}

\begin{proof}{\ } \\
    ``$\Leftarrow$'': Sei $I \subsetneq J$, dann gibt es ein $a$ mit $a \in I \land a' \in I$, womit $1 \in I$ ist und damit $I$ nicht echt.

    ``$\Rightarrow$'': Angenommen es gäbe ein $a \in B$ mit $a \notin I, a' \notin I$. Betrachte
    $$ J := \{ x \in B \mid \exists y \in I: x \leq y \lor a \}. $$
    Klarerweise gilt $I \subseteq J$. Weiters ist $J$ ein Ordnungsideal nach Definition. Sind $x, \widetilde{x}$ in $J$, dann gibt es $y, \widetilde{y} \in I$ mit $x \leq y \lor a, \widetilde{x} \leq \widetilde{y} \lor a$, womit folgt $x \lor \widetilde{x} \leq (y \lor \widetilde{y}) \lor a$. Weiters ist $1 \notin J$, sonst gäbe es ein $y \in I$ mit $1 = a \lor y$, womit folgen würde $a' = y \land a'$, damit $a' \leq y$ und damit $a' \in I$, im Widerspruch.
\end{proof}

\begin{definition}
    Sei $B$ eine Boolesche Algebra. Wir nennen $F \subseteq B$ einen \emph{Ultrafilter}, wenn $F$ ein maximaler echter Filter ist, was genau dann der Fall ist, wenn $F'$ ein maximales echtes Ideal ist.
\end{definition}

\begin{theorem}
    Jeder echte Filter in einer Booleschen Algebra ist in einem Ultrafilter enthalten.
\end{theorem}

\begin{proof}
    Folgt mit dem Lemma von Zorn.
\end{proof}

\begin{lemma}
    Sei $B$ eine Boolesche Algebra, $\vert B \vert \geq 2$, und $I \subsetneq B$ ein maximales echtes Ideal. Dann ist $B /_I \cong \mathfrak{B}_2$. 
\end{lemma}

\begin{proof}
    Seien $a, b \in B \setminus I$, dann ist $a - b = a + b = (a \land b') \lor (b \land a') \in I$, womit $a + I = b + I$ folgt. Damit ist $\vert B/_I \vert = 2$, womit bereits die Aussage folgt.
\end{proof}

\begin{proof}[Beweis vom Satz von Stone]
    Sei $B$ eine Boolesche Algebra. Definiere
    $$ M := \{ F \subseteq B \mid F \text{ ist Ultrafilter} \}, $$
    $$ f : B \to \mathcal{P}(M), \; x \mapsto \{ F \in M \mid x \in F \}. $$
    \textit{$f$ ist injektiv:} Wir wollen zeigen, dass wenn $x \neq y$, dann gibt es ein $F \in M$ mit $x \in F, y \notin F$ oder andersherum. Dazu unterscheiden wir:

    $ x \leq y $: Wir wählen $F$ mit $x' \land y \in F$. Dies ist möglich, da der einzige Problemfall $x' \land y = 0$ ist. In diesem Fall wäre jedoch $y \lor x = x$, im Widerspruch zu $x \leq y$.

    $ y \leq x $: Analog.

    $ x \nleq y, y \nleq x $: Wähle $F$ wie im ersten Fall.

    \textit{$f$ ist ein Homomorphismus:}
    \begin{itemize}
        \item $f(1) = M$
        \item $f(0) = \emptyset$
        \item $x \in B: f(x') = M \setminus f(x)$
        \item $x, y \in B: f(x \lor y) = \{ F \in M \mid x \lor y \in F \} = \{ F \in M \mid x \in F \lor y \in F \} = f(x) \cup f(y)$
        \item $x, y \in B: f(x \land y) = \{ F \in M \mid x \land y \in F \} = \{ F \in M \mid x \in F \land y \in F \} = f(x) \cap f(y)$
    \end{itemize}

    Sei nun $B$ endlich. Wir bemerken, dass wenn $a$ ein Atom ist, dann ist
    $$ F_a := \{ b \in B \mid b \geq a \} $$
    ein Ultrafilter, da wenn $b \notin F_a$, also wenn $b \ngeq a$, dann ist $b \land a = 0$ und damit $b' \in F_a$.

    Sei $x \in B$, dann ist
    $$ f(x) = \{ F_a \mid a \leq x \}. $$
    Jedes $F \in M$ ist von der Form $F_a$, indem wir $a := \bigcap F$ setzen.

    \textit{$f$ ist surjektiv:} Seien $F_{a_1}, \hdots, F_{a_n} \in M$ beliebig. Wähle $x := a_1 \lor \hdots \lor a_n$. Dann gilt zunächst sicher $F_{a_1}, \hdots, F_{a_n} \in f(x)$. Sei angenommen $a$ ist ein Atom und $F_a \in f(x)$, dann ist $a \leq x = a_1 \lor \hdots \lor a_n$. Schneiden mit $a$ liefert $a \leq (a_1 \land a) \lor (a_2 \land a) \lor \hdots (a_n \land a)$. Falls für alle $i$ gilt $a_i \neq a$ ist damit $a \leq 0 \lor \hdots \lor 0$, im Widerspruch.
\end{proof}