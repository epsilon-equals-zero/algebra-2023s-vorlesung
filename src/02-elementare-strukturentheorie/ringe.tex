
\notedate{03.05.2023}

\section{Ringe}

Zu Beginn dieses Abschnitts sei an \cref{def:ring} eines \emph{Rings} erinnert.

\begin{example} Ringe sind unter anderen
    \begin{itemize}
        \item der kommutative Ring mit 1 der ganzen Zahlen $(\mathbb{Z}, +, 0, -, \cdot, 1)$,
        \item der kommutative Ring mit 1 der reellen Polynomfunktionen $(P, +, 0, - \cdot, 1)$, wobei $P \subseteq \mathbb{R}^\mathbb{R}$ die Menge aller Polynomfunktionen ist, $+, \cdot$ punktweise Operationen sind und $0, 1$ konstante Polynome mit entsprechendem Wert,
        \item der (nicht kommutative) Ring mit 1 der reellen $2 \times 2$ Matrizen $(\mathbb{R}^{2 \times 2}, +, (0)_{2 \times 2}, -, \cdot, E_2)$ und
        \item der kommutative Ring $(m\mathbb{Z}, +, 0, -, \cdot), m \geq 2$ der kein Einselement enthält.
    \end{itemize}
\end{example}

\begin{remark}
    Wie auch schon im Abschnitt über Gruppen werden wir im Folgenden für einen Ring $\mathfrak{R} = (R, +, 0, -, \cdot)$ mit Einselement $1$, falls dieses existiert nur $R$ schreiben, also den Ring mit der Trägermenge identifizieren.
\end{remark}

\begin{definition}
    Sei $R$ ein Ring, so heißt $\emptyset \neq I \subseteq R$ \emph{Ideal}\index{Ideal}, oder kurz $I \vartriangleleft R$, genau dann wenn
    \begin{itemize}
        \item $(I, +, 0, -)$ eine Untergruppe von R ist und
        \item $\forall r \in R: rI \subseteq I \land Ir \subseteq I.$
    \end{itemize}
    Gilt bei letzterer Bedingung nur $rI \subseteq I$, beziehungsweise $Ir \subseteq I$, so heißt $I$ \emph{Linksideal}\index{Ideal!Links-}, beziehungsweise \emph{Rechtsideal}\index{Ideal!Rechts-}.
\end{definition}

\begin{remark}
    Ein Ideal $I$ eines Rings $R$ ist definitionsgemäß eine Untergruppe der additiven Gruppe $(R, +, 0, -)$.

    Weiters ist $I$ sogar ein Unterring von $R$, da $I$ nach Definition unter $\cdot$ abgeschlossen ist.
\end{remark}

\begin{remark}\label{remark:ideal-mit-1-trivial}
    Für ein Ideal $I$ eines Rings $R$ gilt $1 \in I \Leftrightarrow I = R$. Nach der Definition ist $I \subseteq R$, für die andere Richtung bemerken wir, dass für alle $r \in R$ gilt $r \cdot 1 = r \in I$.
\end{remark}

\begin{example}
    Betrachte den Ring $(\mathbb{Q}, +, 0, -, \cdot, 1)$, so ist $\mathbb{Z}$ ein Unterring, jedoch kein Ideal.
\end{example}

\begin{example}
    Es ist $m \mathbb{Z} \subseteq (\mathbb{Z}, +, 0, -, \cdot, 1)$ ein Ideal. Sei $P$ der Ring der reellen Polynomfunktionen. Dann ist $(x^2 + 1) \cdot P \vartriangleleft P$. Dies ist ein allgemeines Prinzip, wie wir später noch sehen werden.

    Sei $M$ eine Menge und betrachte den Ring $(\mathcal{P}(M), \triangle, \emptyset, \id_{\mathcal{P}(M)}, \cap, M)$. Sei $A \subseteq M$ beliebig, so ist $\mathcal{P}(A) \vartriangleleft \mathcal{P}(M)$. Weiters kann $(\mathcal{P}(A), \triangle, \emptyset, \id_{\mathcal{P}(A)}, \cap, A)$ zu einem Ring mit 1 gemacht werden. Es handelt sich dabei um keinen Widerspruch zu \cref{remark:ideal-mit-1-trivial}, da hier ein anders Einselement gefunden wird als im ursprünglichen Ring.
\end{example}

\begin{remark}\label{remark:ideal-congrel}
    Sei $(R, +, 0, -, \cdot)$ ein Ring und $\sim \subseteq R^2$ eine Kongruenzrelation auf $R$. Dann ist $\sim$ insbesondere eine Kongruenzrelation auf $(R, +, 0, -)$, womit $\sim$ eindeutig durch $[0]_{\sim}$ bestimmt ist.

    Sind $x, y \in R$ beliebig, $x, y \in [0]_{\sim}$, so gilt $x + y \in [0]_{\sim}, (-x) \in [0]_{\sim}$, vergleiche die Theorie von Normalteilern von Gruppen. Sei $r \in R$ beliebig, so gilt $x \sim 0, r \sim r$, und da $\sim$ Kongruenzrelation ist damit $r \cdot x \sim 0 \cdot r = 0$, also folgt $[0]_{\sim} \vartriangleleft R$.

    Umgekehrt sei $I \vartriangleleft R$ ein Ideal, wir wollen eine entsprechende Kongruenzrelation $\sim$ definieren. Für $x, y \in R$ definieren wir
    $$ x \sim y :\Leftrightarrow y - x \in I. $$
    Wir wissen, dass $\sim$ eine Kongruenzrelation bezüglich $(R, +, 0, -)$ ist. Sei $a \sim b, c \sim d$, dann folgt
    $$ (a-b) \cdot d \in I, \quad a \cdot (c - d) \in I \implies (a-b)\cdot d + a\cdot(c-d) \in I. $$
    Letzerer Ausdruck ist jedoch gleich
    $$ ad - bd + ac - ad = -(bd - ac), $$
    also folgt $ac \sim bd$ und $\sim$ ist auch eine Kongruenzrelation bezüglich $\cdot$.
\end{remark}

\begin{definition}
    Sei $R$ ein Ring, $I \vartriangleleft G$ ein Ideal, dann definieren wir für $a \in R$ die \emph{Nebenklasse von a modulo I}\index{Nebenklasse} als 
    $$a+I := \{ a + r \mid r \in I \}.$$
\end{definition}

\begin{definition}
    Sei $R$ eine Ring, $I \vartriangleleft G$ ein Ideal und $\sim$ die wie in \cref{remark:ideal-congrel} vom Ideal induzierte Kongruenzrelation. Wir definieren den \emph{Faktorring} \index{Ring!Faktor-}
    $$ R/_I := R/_\sim = \{ a + I \mid a \in R\}.$$
    Dabei ist
    $$ (a + I) + (b + I) := (a + b) + I \quad\text{und}\quad (a + I)\cdot(b+I) = (a\cdot b) + I. $$
\end{definition}

\begin{definition}
    Sei $R$ ein Ring, $A \subseteq R$, $a \in R$, so heißen
    $$ (A) := \bigcap \{ I \vartriangleleft R \mid A \subseteq I \}, $$
    $$ (a) := \bigcap \{ I \vartriangleleft R \mid a \in I \} $$
    die von $A$, beziehungsweise $a$, \emph{erzeugten Ideale}\index{erzeugtes Ideal}.
\end{definition}

\begin{remark}
    Man beachte dass $(A)$ und $(a)$ tatsächlich Ideale sind, da Ideale unter Schnitten abgeschlossen sind.
\end{remark}

\begin{remark}
    Wir bemerken, dass gilt
    $$ (A) = \left\{ \sum_i r_i a_i s_i + \sum_j r_j' a_j' + \sum_k a_k'' s_k'' + \sum_\ell a_\ell''' \mid a_i, a_j', a_k'' \in A, a_\ell''' \in A \cup (-A), r_i, r_j', s_i, s_k'' \in R \right\}. $$

    Ist $R$ sogar ein kommutativer Ring mit 1, so gilt
    $$ (A) = \left\{ \sum_i r_i a_i \mid r_i \in R, a_i \in A \right\}. $$
\end{remark}

\begin{definition}
    Sei $R$ ein Ring. Wir nennen $I \vartriangleleft R$ \emph{Hauptideal}\index{Hauptideal}, wenn gilt
    $$ \exists a \in R: I = (a). $$

    Weiters nennen wir $R$ einen \emph{Hauptidealring}\index{Hauptideal!-ring}, wenn gilt
    $$ \forall I \vartriangleleft R : I \text{ ist Hauptideal}. $$
\end{definition}

\begin{example}
    Es ist $(\mathbb{Z}, +, 0, -, \cdot, 1)$ ein Hauptidealring, da alle Unterringe von der Form $m \mathbb{Z} = (m)$ sind.
\end{example}

\begin{definition}
    Ein Ring $R$ heißt \emph{nullteilerfrei}\index{Ring!nullteilerfrei}, wenn
    $$ \forall a, b \in R: (a \cdot b = 0 \Rightarrow a = 0 \lor b = 0) $$

    Ist $R$ ein kommutativer Ring mit 1 und nullteilerfrei, so nennen wir $R$ \emph{Integritätsbereich}\index{Integrationsbereich}.
\end{definition}

\begin{example}
    Ist $R$ ein Körper, so ist $R$ nullteilerfrei, da mit $0 \neq a \in R, b \in R$ gilt
    $$ a b = 0 \Rightarrow b = a^{-1} a b = a^{-1} 0 = 0. $$
\end{example}

\begin{example}
    Es ist $(\mathbb{Z}, +, 0, -, \cdot, 1)$ ein Integritätsbereich, jedoch kein Körper.
\end{example}

\begin{proposition}
    Ist $R$ ein Integritätsbereich und endlich, so ist $R$ ein Körper.
\end{proposition}

\begin{proof}
    Sei $r \in R \setminus \{0\}$, wir wollen ein multiplikatives Inverses finden. Betrachte die Abbildung
    $$ \varphi_r : R \to R, x \mapsto r \cdot x. $$
    $\varphi_r$ ist injektiv: Sei $\varphi_r(x) = \varphi_r(y)$, so folgt $rx = ry$, also $r(x-y) = 0$, also $x-y=0$, also $x=y$.
    Da $R$ endlich ist, ist damit $\varphi_r$ auch surjektiv, also gibt es ein $x \in R$ mit $\varphi_r(x) = r \cdot x = 1$.
\end{proof}

\begin{proposition}
    Sei $R$ ein kommutativer Ring mit 1, dann ist $R$ ein Körper genau dann, wenn
    $$ \forall I \vartriangleleft R: (I = \{0\} \lor I = R). $$
\end{proposition}

\begin{proof}{\ }
    \begin{itemize}
        \item[$\Rightarrow$:] Sei $I \neq \{0\}, x \in I, x \neq 0$, so ist $1 = x^{-1} x \in I$, also $I = R$.
        \item[$\Leftarrow$:] Sei $R$ kein Körper, so gibt es ein $x \in R \setminus \{0\}$ sodass für alle $y \in R$ gilt $xy \neq 1$. Setze $I := (x) \vartriangleleft R$, so gilt wegen $x \in I$ dass $I \neq \{0\}$. Wegen $1 \notin I$ ist auch $I \neq R$.
    \end{itemize}
\end{proof}

\begin{definition}
    Sei $I \vartriangleleft R$. Wir nennen $I$
    
    \begin{itemize}
        \item \emph{echt}\index{Ideal!echt}, wenn $I \subsetneq R$, 
        \item \emph{prim}\index{Ideal!prim}, wenn $I$ echt ist und
        $ \forall a, b \in R: (ab \in I \Rightarrow a \in I \lor b \in I) $ und
        \item\emph{maximal}\index{Ideal!maximal}, wenn $I$ echt ist und
        $\forall J \vartriangleleft R : J \supsetneq I \Rightarrow J = R. $
    \end{itemize}
\end{definition}

\begin{example}
    Sei $p \in \mathbb{P}$, so ist $p \mathbb{Z} \vartriangleleft \mathbb{Z}$ prim. Ist $m \in \mathbb{N}_{\geq 2} \setminus \mathbb{P}$, so ist $m \mathbb{Z}$ nicht prim.
\end{example}

\begin{proposition}
    Sei $R$ ein kommutativer Ring mit 1 und $I \vartriangleleft R$. Dann gilt:
    \begin{itemize}
        \item $R /_{I} \text{ ist Körper} \Leftrightarrow I \text{ ist maximal}$
        \item $R /_{I} \text{ ist prim} \Leftrightarrow I \text{ ist prim}$
        \item $I \text{ ist maximal} \Rightarrow I \text{ ist prim}$
        \item $I \text{ ist echt} \Rightarrow \exists J \supseteq I: J \vartriangleleft R \text{ ist maximal}$
    \end{itemize}
\end{proposition}

\begin{proof}{\ }
    \begin{enumerate}
        \item \begin{itemize}
            \item[$\Rightarrow$:] Angenommen $I$ wäre nicht maximal, es gibt also ein $R \neq J \supsetneq I, J \vartriangleleft R$.
            Sei $J' := \{ a + I \mid a \in J \}$. Dann ist $J' \vartriangleleft R/_I, J' \neq R/_I$ und $J \neq \{I\}$. Also hat $R/_I$ ein echtes Ideal, im Widerspruch dazu, dass $R/_I$ ein Körper ist.
            \item[$\Leftarrow$:] Sei $I$ maximal. Wir behaupten, dass $R/_I$ keine echten Ideale außer dem trivialen hat. Wäre dies nicht so, so sei $J \vartriangleleft R/_I$ echt, $J \neq \{I\}$ und sei $J' := \bigcup_{M \in J} M$. Dann ist $J' \supsetneq I, J' \neq R, J' \vartriangleleft R$, im Widerspruch zur Maximalität von $I$.
        \end{itemize}
        \item Es gilt
        \begin{align*}
            R/_I \text{ ist Integritätsbereich} &\Leftrightarrow \forall a, b \in R: (a + I)(b + I) = I \Rightarrow a + I = I \lor b + I = I \\
            &\Leftrightarrow \forall a, b \in R: ab \in I \Rightarrow a \in I \lor b \in I \\
            &\Leftrightarrow I \text{ ist prim}.
        \end{align*}
        \item Folgt direkt aus (1) und (2).
        \item Folgt aus dem Lemma von Zorn.
    \end{enumerate}
\end{proof}