
\notedate{03.05.2023}

\section{Ringe}

Es sei an \cref{def:ring} eines Rings erinnert.

\begin{example}
    Der kommutative 1-Ring der ganzen Zahlen $(\mathbb{Z}, +, 0, -, \cdot, 1)$.
\end{example}

\begin{example}
    Es sei $P \subseteq \mathbb{R}^{\mathbb{R}}$ die Menge der Polynomfunktionen, so ist $(P, +, 0, -, \cdot, 1)$ ein Ring mit 1.
\end{example}

\begin{example}
    Der 1-Ring der reelen $2 \times 2$ Matrizen $(\mathbb{R}^{2 \times 2}, +, (0)_{2 \times 2}, -, \cdot, E_2)$.
\end{example}

\begin{example}
    Der Ring $(m\mathbb{Z}, +, 0, -, \cdot), m \geq 2$ besitzt kein Einselement.
\end{example}

\begin{definition}
    Sei $R$ ein Ring, so heißt $\emptyset \neq I \subseteq R$ Ideal genau dann wenn
    \begin{itemize}
        \item $(I, +, 0, -)$ ist eine Untergruppe von R,
        \item $\forall r \in R: rI \subseteq I \land Ir \subseteq I$
    \end{itemize}
    Gilt bei letzterer Bedingung nur $rI \subseteq I$, beziehungsweise $Ir \subseteq I$, so heißt $I$ Linksideal, beziehungsweise Rechtsideal.

    Ist $I$ ein Ideal von $R$ so schreiben wir auch $I \vartriangleleft R$.
\end{definition}

\begin{remark}
    Sei $I$ ein Ideal. Wir bemerken, dass dann $I$ eine Untergruppe von $(R, +, 0, -)$ ist. Nach Definition ist $I$ abgeschlossen unter $+$ und für $x \in I$ gilt $(-1) \cdot x = -x \in I$.

    Weiters ist $I$ sogar ein Unterring von $R$, da $I$ nach Definition unter $\cdot$ abgeschlossen ist.
\end{remark}

\begin{remark}
    Sei $I$ ein Ideal. Wir bemerken, dass $1 \in I \Leftrightarrow I = R$. Es gilt $I \subseteq R$ nach Definition, für die andere Richtung bemerken wir, dass für alle $r \in R$ gilt $r \cdot 1 = r \in I$.
\end{remark}

\begin{example}
    Betrachte den Ring $(\mathbb{Q}, +, 0, -, \cdot, 1)$, so ist $\mathbb{Z}$ ein Unterring, jedoch kein Ideal.
\end{example}

\begin{example}
    Es ist $m \mathbb{Z} \subseteq (\mathbb{Z}, +, 0, -, \cdot, 1)$ ein Ideal.
\end{example}

\begin{example}
    Sei $M$ eine Menge und betrachte den Ring $(\mathcal{P}(M), \triangle, \emptyset, \id_{\mathcal{P}(M)}, \cap, M)$. Sei $A \subseteq M$ beliebig, so ist $\mathcal{P}(A) \vartriangleleft \mathcal{P}(M)$. Weiters ist $(\mathcal{P}(A), \triangle, \emptyset, \id_{\mathcal{P}(A)}, \cap, A)$ ein weiterer Ring mit 1.
\end{example}

\begin{example}
    Sei wieder $P$ die Menge der Polynomfunktionen von $\mathbb{R}$ in sich. Dann ist $(x^2 + 1) \cdot P \vartriangleleft P$. Dies ist ein allgemeines Prinzip, wie wir später noch sehen werden.
\end{example}

\begin{remark}
    Sei $(R, +, 0, -, \cdot)$ ein Ring und $\sim \subseteq R^2$ eine Kongruenzrelation auf $R$. Dann ist $\sim$ insbesondere eine Kongruenzrelation auf $(R, +, 0, -)$, womit $\sim$ eindeutig durch $[0]_{\sim}$ bestimmt ist.

    Sind $x, y \in R$ beliebig, $x, y \in [0]_{\sim}$, so gilt $x + y \in [0]_{\sim}, (-x) \in [0]_{\sim}$, vergleiche die Theorie von Normalteilern von Gruppen. Sei $r \in R$ beliebig, so gilt $x \sim 0, r \sim r$, und da $\sim$ Kongruenzrelation ist damit $r \cdot x \sim 0 \cdot r = 0$, also folgt $[0]_{\sim} \vartriangleleft R$.

    Umgekehrt sei $I \vartriangleleft R$ ein Ideal, wir wollen eine entsprechende Kongruenzrelation $\sim$ definieren. Für $x, y \in R$ definieren wir
    $$ x \sim y :\Leftrightarrow y - x \in I. $$
    Wir wissen, dass $\sim$ eine Kongruenzrelation bezüglich $(R, +, 0, -)$ ist. Sei $a \sim b, c \sim d$, dann folgt
    $$ (a-b) \cdot d \in I, \quad a \cdot (c - d) \in I \implies (a-b)\cdot d + a\cdot(c-d) \in I. $$
    Letzerer Ausdruck ist jedoch gleich
    $$ ad - bd + ac - ad = -(bd - ac), $$
    also folgt $ac \sim bd$ und $\sim$ ist auch eine Kongruenzrelation bezüglich $\cdot$.
\end{remark}

\begin{definition}
    Sei $R$ ein Ring, $A \subseteq R$, $a \in R$, so heißen
    $$ (A) := \bigcap \{ I \vartriangleleft R \mid A \subseteq I \}, $$
    $$ (a) := \bigcap \{ I \vartriangleleft R \mid a \in I \} $$
    die von $A$, beziehungsweise $a$, \emph{erzeugten Ideale}.
\end{definition}

\begin{remark}
    Man beachte dass $(A)$ und $(a)$ tatsächlich Ideale sind, da Ideale unter Schnitten abgeschlossen sind.
\end{remark}

\begin{remark}
    Wir bemerken, dass gilt
    $$ (A) = \left\{ \sum_i r_i a_i s_i + \sum_j r_j' a_j' + \sum_k a_k'' s_k'' + \sum_\ell a_\ell''' \mid a_i, a_j', a_k'' \in A, a_\ell''' \in A \cup (-A), r_i, r_j', s_i, s_k'' \in R \right\}. $$

    Ist $R$ sogar ein kommutativer Ring mit 1, so gilt
    $$ (A) = \left\{ \sum_i r_i a_i \mid r_i \in R, a_i \in A \right\}. $$
\end{remark}

\begin{definition}
    Sei $R$ ein Ring. Wir nennen $I \vartriangleleft R$ \emph{Hauptideal}, wenn gilt
    $$ \exists a \in R: I = (a). $$

    Weiters nennen wir $R$ einen \emph{Hauptidealring}, wenn gilt
    $$ \forall I \vartriangleleft R : I \text{ ist Hauptideal}. $$
\end{definition}

\begin{example}
    Es ist $(\mathbb{Z}, +, 0, -, \cdot, 1)$ ein Hauptidealring, da alle Untergruppen von der Form $m \mathbb{Z} = (m)$ sind.
\end{example}

\begin{definition}
    Ein Ring $R$ heißt \emph{nullteilerfrei}, wenn
    $$ \forall a, b \in R: (a \cdot b = 0 \implies a = 0 \lor b = 0) $$
\end{definition}

\begin{example}
    Ist $R$ ein Körper, so ist $R$ nullteilerfrei, da mit $0 \neq a \in R, b \in R$ gilt
    $$ a b = 0 \implies b = a^{-1} a b = a^{-1} 0 = 0. $$
\end{example}

\begin{definition}
    Ist $R$ ein kommutativer Ring mit 1 und nullteilerfrei, so nennen wir $R$ \emph{Integritätsbereich}.
\end{definition}

\begin{example}
    Es ist $(\mathbb{Z}, +, 0, -, \cdot, 1)$ ein Integritätsbereich, jedoch kein Körper.
\end{example}

\begin{proposition}
    Ist $R$ ein Integritätsbereich und endlich, so ist $R$ ein Körper.
\end{proposition}

\begin{proof}
    Sei $r \in R \setminus \{0\}$, wir wollen ein multiplikatives Inverses finden. Betrachte die Abbildung
    $$ \varphi_r : R \to R, x \mapsto r \cdot x. $$
    $\varphi_r$ ist injektiv: Sei $\varphi_r(x) = \varphi_r(y)$, so folgt $rx = ry$, also $r(x-y) = 0$, also $x-y=0$, also $x=y$.
    Da $R$ endlich ist, ist damit $\varphi_r$ schon surjektiv, also gibt es ein $x \in R$ mit $\varphi_r(x) = r \cdot x = 1$.
\end{proof}

\begin{proposition}
    Sei $R$ ein kommutativer Ring mit 1, dann ist $R$ ein Körper genau dann, wenn
    $$ \forall I \vartriangleleft R: (I = \{0\} \lor I = R). $$
\end{proposition}

\begin{proof}{\ }
    \begin{itemize}
        \item $\implies$: Sei $I \neq \{0\}, x \in I, x \neq 0$, so ist $1 = x^{-1} x \in I$, also $I = R$.
        \item $\impliedby$: Sei $R$ ein kein Körper, so gibt es ein $x \in R \setminus \{0\}$ sodass für alle $y \in R$ gilt $xy \neq 1$. Setze $I := (x) \vartriangleleft R$, so gilt wegen $x \in I$ dass $I \neq \{0\}$. Wegen $1 \notin I$ ist auch $I \neq R$.
    \end{itemize}
\end{proof}

\begin{definition}
    Sei $I \vartriangleleft R$. Wir nennen $I$ \emph{prim}, wenn
    $$ \forall a, b \in R: (ab \in I \implies a \in I \lor b \in I). $$
    Weiters nennen wir $I$ \emph{maximal}, wenn
    $$ I \subsetneq R \land \forall J \vartriangleleft R : J \supsetneq I. $$
\end{definition}

\begin{example}
    Sei $p \in \mathbb{P}$, so ist $p \mathbb{Z} \vartriangleleft \mathbb{Z}$ prim. Ist $m \in \mathbb{N}_{\geq 2} \setminus \mathbb{P}$, so ist $m \mathbb{Z}$ nicht prim.
\end{example}

\begin{proposition}
    Sei $R$ ein kommutativer Ring mit 1 und $I \vartriangleleft R$. Dann gilt:
    \begin{itemize}
        \item $R /_{I} \text{ ist Körper} \Leftrightarrow I \text{ ist maximal}$
        \item $R /_{I} \text{ ist prim} \Leftrightarrow I \text{ ist prim}$
        \item $I \text{ ist maximal} \implies I \text{ ist prim}$
        \item $I \text{ ist echt} \implies \exists J \supseteq I: J \vartriangleleft R \text{ ist maximal}$
    \end{itemize}
\end{proposition}

\begin{proof}{\ }
    \begin{enumerate}
        \item \begin{itemize}
            \item $\implies$: Angenommen $I$ wäre nicht maximal, es gibt also ein $J \supsetneq I, J \vartriangleleft R$.
            Sei $J' := \{ a + I \mid a \in J \}$. Dann ist $J' \vartriangleleft R/_I, J' \neq R/_I$, also ist $J' \neq \{I\}$. Also hat $R/_I$ ein echtes Ideal, im Widerspruch dazu, dass $R/_I$ ein Körper ist.
            \item $\impliedby$: Sei $I$ maximal. Wir behaupten, dass $R/_I$ keine echten Ideale außer dem trivialen hat. Wäre dies nicht so, so sei $J \vartriangleleft R/_I$ echt, $J \neq \{I\}$ und sei $J' := \bigcup J$. Dann ist $J' \supsetneq I, J' \neq R, J' \vartriangleleft R$, im Widerspruch zur Maximalität von $I$.
        \end{itemize}
        \item Es gilt
        \begin{align*}
            R/_I \text{ ist Integritätsbereich} &\Leftrightarrow \forall a, b \in R: (a + I)(b + I) = I \implies a + I = I \lor b + I = I \\
            &\Leftrightarrow \forall a, b \in R: ab \in I \implies a \in I \lor b \in I \\
            &\Leftrightarrow I \text{ ist prim}.
        \end{align*}
        \item Folgt direkt aus (1) und (2).
        \item Folgt aus dem Lemma von Zorn.
    \end{enumerate}
\end{proof}

\notedate{04.05.2023}

\begin{example}
    Betrachte den Ring $\mathbb{Z}$, $p \in \mathbb{P}$ und $p \mathbb{Z} \vartriangleleft \mathbb{Z}$, so erhalten wir $\mathbb{Z}/_{p\mathbb{Z}} = \mathbb{Z}/_p$.

    Für ein $m \in \mathbb{N}$ betrachte $m \mathbb{Z} \vartriangleleft \mathbb{Z}$, so ist zwar $\mathbb{Z} /_{m\mathbb{Z}} = \mathbb{Z}_m$, jedoch kein Körper.
\end{example}

\begin{example}
    Sei $P$ die Menge der Polynomfunktionen auf $\mathbb{R}$ und $(x^2+1) \cdot P \vartriangleleft P$, so ist dies ein Primideal, und $P/_{(x^2+1) \cdot P}$ sogar ein Integritätsbereich. Jedoch ist es kein Körper, da $(x^2+1) \cdot P$ nicht maximal ist -- betrachte dazu beispielsweise
    $$ (x^2 + 1) \cdot P \subsetneq (x^2+1) \cdot P + x \cdot P \vartriangleleft P $$.

    Das Ideal $I := (x^2 - 1) \cdot P \vartriangleleft P$ ist kein Primideal, da $(x-1)\notin I, (x+1) \notin I$, aber $(x-1)(x+1) = x^2 - 1 \in I$.
\end{example}

\begin{definition}
    Wir definieren die $\emph{Charakteristik}$ eines Rings als
    $$ \mathrm{char}{R} := $$
    TODO
\end{definition}

\begin{proposition}
    Sei $R$ ein Ring mit 1, $\mathrm{char} R = p \in \mathbb{P}$. Dann ist
    $$ \varphi : R \to R, x \mapsto x^{p^k} $$
    ein Homomorphismus.
\end{proposition}

\begin{proof}
    Wir zeigen die Aussage mittels Induktion nach $k$.

    Induktionsanfang ($k = 1$): Es gilt
    $$ (a+b)^p = \sum_{i=0}^p {p \choose i} a^i b^{p-1}. $$
    Wir beobachten
    $$ {p \choose i} = \frac{p \cdot (p-1) \cdot \hdots \cdot (p-i+1)}{} \equiv 0 mod p $$
    TODO
\end{proof}

\begin{remark}
    Sei $R$ ein kommutativer Ring mit 1, wir wollen $R$ in einen Körper einbetten. Es gelte \obda $0 \neq 1$ (da sonst für alle $x$ gilt $ x = 1\cdot x = 0 \cdot x = 0$). Wir sammeln nun notwendige Voraussetzungen.

    Es muss $R$ ein Integritätsbereich sein, da sonst
    TODO

    Nicht notwendig (da es aus der vorigen Bedingung folgt), aber interessant ist die Tatsache, dass wenn $R$ ein Integritätsbereich ist, jedes Element außer $0$ bereits kürzbar ist.
    TODO

    Es ist $R^* := R \setminus \{0\}$ ein kommutatives Monoid mit $\cdot$. Definiere eine Äquivalenzrelation $\sim \subseteq (R \times R^*)^2, (a, b) \sim (c, d) :\Leftrightarrow ad = bc$. Wie man nachrechnet ist dies sogar eine Kongruenzrelation. Dann ist $((R \times R^*)/_\sim, \cdot, [(1,1])_\sim) := M$ ein Monoid, wobei jedes $[(x,y)]_\sim$ mit $x \neq 0$ ein Inverses besitzt.
    
    Dann ist
    $$ \varphi : R \to M, x \mapsto [(x, 1)]_\sim $$
    eine Einbettung von $R$ in $M$.

    Für jedes multiplikative Monoid $N$ mit einer Einbettung $\psi : R \to N$ gibt es eine Einbettung $\bar{\psi} : M \to N$ mit $\bar{\psi} \circ \varphi = \psi$.

    Auf $M$ definieren wir nun eine Addition
    $$ (a,b) + (c,d) := (ad+bc, bd), \quad -(a,b) := (-a, b), $$
    so ist $(R \times R^*, +, (0,1), -)$ eine Gruppe.
\end{remark}

\begin{lemma}
    Die wie oben definierte Äquivalenzrelation $\sim \subset (R \times R^*)^2$ ist eine Kongruenzrelation bezüglich $+$.
\end{lemma}

\begin{proof}
    nachrechnen TODO
\end{proof}

\begin{theorem}
    Sei $R$ ein kommutativer Ring mit $1 \neq 0$, der zusätzlich ein Integritätsbereich ist. Sei weiters $\sim \subseteq (R \times R^*)^2$ wie oben definiert. Dann gilt:
    \begin{enumerate}
        \item $K := (R \times R^*)^2/_\sim $ mit $+, \cdot$ wie oben definiert ist ein Körper.
        \item $\varphi : R \to K, x \mapsto [(x, 1)]_\sim$ ist eine Einbettung.
        \item Für alle Einbettungen $\psi : R \to L$ in einen Körper $L$ gibt es eine Einbettung $\bar{\psi} : K \to L$ mit $\bar{\psi} \circ \varphi = \psi$.
    \end{enumerate}
\end{theorem}

\begin{proof}
    Wir haben bereits gezeigt, dass $\sim$ eine Kongruenzrelation ist, womit $K$ wohldefiniert ist.

    Wir wissen $(K, \cdot, [(1,1)]_\sim)$ ist ein kommutatives Monoid.

    Weiters ist $(K \setminus \{ [(0,1)]_\sim \}, \cdot, [(1, 1)]_\sim) = (R^* \times R^*)/_\sim$ eine kommutative Gruppe, genauso auch $(K, +, [(0,1)]_\sim, -)$.

    Das Distributivgesetz verifiziert man unmittelbar durch Nachrechnen.

    Nach Konstruktion ist $\varphi$ eine injektive Einbettung bezüglich $\cdot$. Wie man nachrechnet ist $\varphi$ sogar eine Einbettung bezüglich $+$.

    Sei $\psi : R \to L$ eine Einbettung in einen Körper $L$. Nach der Monoidkonstruktion gibt es eine Einbettung $\bar{\psi} : K \to L$ bezüglich $\cdot$ mit $\bar{\psi} \circ \varphi = \psi$. Wir verifizieren nun
    \begin{align*}
        \bar{\psi} ([(z_1, u_1)]_\sim + [(z_2, u_2)]_\sim) &= \bar{\psi} ([(z_1 u_2 + z_2 u_1, u_1 u_2)]_\sim) \\
        &= \bar{\psi}([])
    \end{align*}
\end{proof}

\begin{proposition}
    Sei $L$ ein Körper mit der obigen Eigenschaft (3), so gilt bereits $L \cong K$, wobei $K$ unser konstruierter Körper.
\end{proposition}

\begin{proof}
    
\end{proof}

\begin{example}
    Sei $R$ ein kommutativer Ring mit 1, $x$ eine Variable und definiere
    $$ R[x] := \{ t(x) \mid t \text{ Term über $x$ in Sprache } +, \cdot, (r)_{r \in R} \}/_\sim, $$
    wobei $\sim$ die von Gesetzen der kommutativen Ringe mit 1 und Gesetzen in $R$ erzeugte Äquivalenzrelation ist. In $R[x]$ gilt also beispielsweise
    $$ x + x \cdot x = x \cdot x + x, \quad r \cdot (s \cdot x) = (r \cdot s) \cdot x. $$
\end{example}