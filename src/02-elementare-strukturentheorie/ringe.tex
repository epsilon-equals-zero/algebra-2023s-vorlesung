
\notedate{03.05.2023}

\section{Ringe}

Zu Beginn dieses Abschnitts sei an \cref{def:ring} eines \emph{Rings} erinnert.

\begin{example} Ringe sind unter anderem
    \begin{itemize}
        \item der kommutative Ring mit 1 der ganzen Zahlen $(\mathbb{Z}, +, 0, -, \cdot, 1)$,
        \item der kommutative Ring mit 1 der reellen Polynomfunktionen $(P, +, 0, -, \cdot, 1)$, wobei $P \subseteq \mathbb{R}^\mathbb{R}$ die Menge aller Polynomfunktionen ist, $+, \cdot$ punktweise Operationen sind und $0, 1$ konstante Polynome mit entsprechendem Wert,
        \item der (nicht kommutative) Ring mit 1 der reellen $2 \times 2$ Matrizen $(\mathbb{R}^{2 \times 2}, +, (0)_{2 \times 2}, -, \cdot, E_2)$ und
        \item der kommutative Ring $(m\mathbb{Z}, +, 0, -, \cdot), m \geq 2$ der kein Einselement enthält.
    \end{itemize}
\end{example}

\begin{remark}
    Wie auch schon im Abschnitt über Gruppen werden wir im Folgenden für einen Ring $\mathfrak{R} = (R, +, 0, -, \cdot)$ nur $R$ schreiben, also den Ring mit der Trägermenge identifizieren.
\end{remark}

\begin{definition}
    Sei $R$ ein Ring, so heißt $\emptyset \neq I \subseteq R$ \emph{Ideal}\index{Ideal}, oder kurz $I \vartriangleleft R$, genau dann wenn
    \begin{itemize}
        \item $(I, +, 0, -)$ eine Untergruppe von R ist und
        \item $\forall r \in R: rI \subseteq I \land Ir \subseteq I.$
    \end{itemize}
    Gilt bei letzterer Bedingung nur $rI \subseteq I$, beziehungsweise $Ir \subseteq I$, so heißt $I$ \emph{Linksideal}\index{Ideal!Links-}, beziehungsweise \emph{Rechtsideal}\index{Ideal!Rechts-}.
\end{definition}

\begin{remark}
    Ein Ideal $I$ eines Ringes $R$ ist ein Unterring von $R$, da $I$ nach Definition unter der Multiplikation abgeschlossen ist.
\end{remark}

\begin{remark}\label{remark:ideal-mit-1-trivial}
    Für ein Ideal $I$ eines Rings $R$ gilt $1 \in I \Leftrightarrow I = R$. Nach der Definition ist $I \subseteq R$, für die andere Richtung bemerken wir, dass für alle $r \in R$ gilt $r \cdot 1 = r \in I$.
\end{remark}

\begin{example}
    Betrachte den Ring $(\mathbb{Q}, +, 0, -, \cdot, 1)$, so ist $\mathbb{Z}$ ein Unterring, jedoch kein Ideal.
\end{example}

\begin{example}
    Es ist $m \mathbb{Z} \subseteq (\mathbb{Z}, +, 0, -, \cdot, 1)$ ein Ideal. Sei $P$ der Ring der reellen Polynomfunktionen. Dann ist $(x^2 + 1) \cdot P \vartriangleleft P$. Dies ist ein allgemeines Prinzip, wie wir später noch sehen werden.

    Sei $M$ eine Menge und betrachte den Ring $(\mathcal{P}(M), \triangle, \emptyset, \id_{\mathcal{P}(M)}, \cap, M)$. Sei $A \subseteq M$ beliebig, so ist $\mathcal{P}(A) \vartriangleleft \mathcal{P}(M)$. Weiters ist $(\mathcal{P}(A), \triangle, \emptyset, \id_{\mathcal{P}(A)}, \cap, A)$ ein Ring mit Einselement. Es handelt sich dabei um keinen Widerspruch zu \cref{remark:ideal-mit-1-trivial}, da hier ein anderes Einselement gefunden wird als im ursprünglichen Ring.
\end{example}

\begin{remark}\label{remark:ideal-congrel}
    Sei $(R, +, 0, -, \cdot)$ ein Ring und $\sim \subseteq R^2$ eine Kongruenzrelation auf $R$. Dann ist $\sim$ insbesondere eine Kongruenzrelation auf $(R, +, 0, -)$, womit $\sim$ eindeutig durch $[0]_{\sim}$ bestimmt ist.

    Sind $x, y \in [0]_{\sim}$, so gilt $x + y \in [0]_{\sim}, (-x) \in [0]_{\sim}$, vergleiche die Theorie von Normalteilern von Gruppen. Sei $r \in R$ beliebig, so gilt $x \sim 0, r \sim r$, und da $\sim$ Kongruenzrelation ist damit $r \cdot x \sim 0 \cdot r = 0$, also folgt $[0]_{\sim} \vartriangleleft R$.

    Umgekehrt sei $I \vartriangleleft R$ ein Ideal, wir wollen eine entsprechende Kongruenzrelation $\sim$ definieren. Für $x, y \in R$ definieren wir
    $$ x \sim y :\Leftrightarrow y - x \in I. $$
    Wir wissen, dass $\sim$ eine Kongruenzrelation bezüglich $(R, +, 0, -)$ ist. Sei $a \sim b, c \sim d$, dann folgt
    $$ (a-b) \cdot d \in I, \quad a \cdot (c - d) \in I \implies (a-b)\cdot d + a\cdot(c-d) \in I. $$
    Letzerer Ausdruck ist jedoch gleich
    $$ ad - bd + ac - ad = -(bd - ac), $$
    also folgt $ac \sim bd$ und $\sim$ ist auch eine Kongruenzrelation bezüglich $\cdot$.
\end{remark}

\begin{definition}
    Sei $R$ ein Ring, $I \vartriangleleft R$ ein Ideal, dann definieren wir für $a \in R$ die \emph{Nebenklasse von a modulo I}\index{Nebenklasse} als 
    $$a+I := \{ a + r \mid r \in I \}.$$
\end{definition}

\begin{definition}
    Sei $R$ eine Ring, $I \vartriangleleft R$ ein Ideal und $\sim$ die wie in \cref{remark:ideal-congrel} vom Ideal $I$ induzierte Kongruenzrelation. Wir definieren den \emph{Faktorring} \index{Ring!Faktor-}
    $$ R/_I := R/_\sim = \{ a + I \mid a \in R\}.$$
    Dabei ist
    $$ (a + I) + (b + I) := (a + b) + I \quad\text{und}\quad (a + I)\cdot(b+I) := (a\cdot b) + I. $$
\end{definition}

\begin{definition}
    Sei $R$ ein Ring, $A \subseteq R$, $a \in R$, so heißen
    $$ (A) := \bigcap \{ I \vartriangleleft R \mid A \subseteq I \}, $$
    $$ (a) := \bigcap \{ I \vartriangleleft R \mid a \in I \} $$
    die von $A$, beziehungsweise $a$, \emph{erzeugten Ideale}\index{erzeugtes Ideal}.
\end{definition}

\begin{remark}
    Man beachte dass $(A)$ und $(a)$ tatsächlich Ideale sind, da Ideale unter Schnitten abgeschlossen sind.
\end{remark}

\begin{remark}\label{remark:darstellung_ideal}
    Wir bemerken, dass gilt
    $$ (A) = \left\{ \sum_i r_i a_i s_i + \sum_j r_j' a_j' + \sum_k a_k'' s_k'' + \sum_\ell a_\ell''' \mid a_i, a_j', a_k'' \in A, a_\ell''' \in A \cup (-A), r_i, r_j', s_i, s_k'' \in R \right\}. $$

    Ist $R$ sogar ein kommutativer Ring mit 1, so gilt
    $$ (A) = \left\{ \sum_i r_i a_i \mid r_i \in R, a_i \in A \right\}. $$
\end{remark}

\begin{example}
    Es ist $(\mathbb{Z}, +, 0, -, \cdot, 1)$ ein Hauptidealring, da alle Unterringe von der Form $m \mathbb{Z} = (m)$ sind.
\end{example}

\begin{definition}
    Ein Ring $R$ heißt \emph{nullteilerfrei}\index{Ring!nullteilerfrei}, wenn
    $$ \forall a, b \in R: (a \cdot b = 0 \Rightarrow a = 0 \lor b = 0) $$

    Ist $R$ ein kommutativer Ring mit 1 und nullteilerfrei, so nennen wir $R$ \emph{Integritätsbereich}\index{Integritätsbereich}.
\end{definition}

\begin{example}
    Ist $R$ ein Körper, so ist $R$ nullteilerfrei, da mit $0 \neq a \in R, b \in R$ gilt
    $$ a b = 0 \Rightarrow b = a^{-1} a b = a^{-1} 0 = 0. $$
\end{example}

\begin{example}
    Es ist $(\mathbb{Z}, +, 0, -, \cdot, 1)$ ein Integritätsbereich, jedoch kein Körper.
\end{example}

\begin{definition}
    Sei $R$ ein Ring. Wir nennen $I \vartriangleleft R$ \emph{Hauptideal}\index{Hauptideal}, wenn gilt
    $$ \exists a \in R: I = (a). $$

    Weiters nennen wir $R$ einen \emph{Hauptidealring}\index{Hauptideal!-ring}\index{Ring!Hauptideal-}, wenn $R$ ein Integritätsbereich ist und
    $$ \forall I \vartriangleleft R : I \text{ ist Hauptideal}. $$
\end{definition}

\begin{proposition}
    Ist $R$ ein Integritätsbereich und endlich, so ist $R$ ein Körper.
\end{proposition}

\begin{proof}
    Sei $r \in R \setminus \{0\}$, wir wollen ein multiplikatives Inverses finden. Betrachte die Abbildung
    $$ \varphi_r : R \to R, x \mapsto r \cdot x. $$
    $\varphi_r$ ist injektiv: Sei $\varphi_r(x) = \varphi_r(y)$, so folgt $rx = ry$, also $r(x-y) = 0$, also $x-y=0$, also $x=y$.
    Da $R$ endlich ist, ist damit $\varphi_r$ auch surjektiv, also gibt es ein $x \in R$ mit $\varphi_r(x) = r \cdot x = 1$.
\end{proof}

\begin{proposition}
    Sei $R$ ein kommutativer Ring mit $1\neq 0$, dann ist $R$ ein Körper genau dann, wenn
    $$ \forall I \vartriangleleft R: (I = \{0\} \lor I = R). $$
\end{proposition}

\begin{proof}{\ }
    \begin{itemize}
        \item[$\Rightarrow$:] Sei $I \neq \{0\}, x \in I, x \neq 0$, so ist $1 = x^{-1} x \in I$, also $I = R$.
        \item[$\Leftarrow$:] Sei $R$ kein Körper, so gibt es ein $x \in R \setminus \{0\}$ sodass für alle $y \in R$ gilt $xy \neq 1$. Setze $I := (x) \vartriangleleft R$, so gilt wegen $x \in I$ dass $I \neq \{0\}$. Wegen $1 \notin I$ ist auch $I \neq R$.
    \end{itemize}
\end{proof}

\begin{corollary}\label{corollary:koerperhomom_injektiv}
    Seien $K, L$ Körper und $\varphi:K\to L$ ein Körperhomomorphismus. Dann ist $\varphi$ injektiv.
\end{corollary}

\begin{proof}
    Es ist $\ker\varphi\triangleleft K$ ein Ideal mit $1\not\in\ker\varphi$.
    Da der Körper $K$ nur die trivialen Ideale hat, folgt $\ker\varphi=\{0\}$ und $\varphi$ ist injektiv.
\end{proof}

\begin{definition}
    Sei $I \vartriangleleft R$. Wir nennen $I$
    
    \begin{itemize}
        \item \emph{echt}\index{Ideal!echt}, wenn $I \subsetneq R$, 
        \item \emph{prim}\index{Ideal!prim}, wenn $I$ echt ist und
        $ \forall a, b \in R: (ab \in I \Rightarrow a \in I \lor b \in I) $ und
        \item\emph{maximal}\index{Ideal!maximal}, wenn $I$ echt ist und
        $\forall J \vartriangleleft R : J \supsetneq I \Rightarrow J = R. $
    \end{itemize}
\end{definition}

\begin{example}
    Sei $p \in \mathbb{P}$, so ist $p \mathbb{Z} \vartriangleleft \mathbb{Z}$ prim. Ist $m \in \mathbb{N}_{\geq 2} \setminus \mathbb{P}$, so ist $m \mathbb{Z}$ nicht prim.
\end{example}

\begin{proposition}\label{prop:ringideale}
    Sei $R$ ein kommutativer Ring mit 1 und $I \vartriangleleft R$. Dann gilt:
    \begin{itemize}
        \item $R /_{I} \text{ ist Körper} \Leftrightarrow I \text{ ist maximal}$
        \item $R /_{I} \text{ ist Integritätsbereich} \Leftrightarrow I \text{ ist prim}$
        \item $I \text{ ist maximal} \Rightarrow I \text{ ist prim}$
        \item $I \text{ ist echt} \Rightarrow \exists J \supseteq I: J \vartriangleleft R \text{ ist maximal}$
    \end{itemize}
\end{proposition}

\begin{proof}{\ }
    \begin{enumerate}
        \item \begin{itemize}
            \item[$\Rightarrow$:] Angenommen $I$ wäre nicht maximal, es gibt also ein $R \neq J \supsetneq I, J \vartriangleleft R$.
            Sei $J' := \{ a + I \mid a \in J \}$. Dann ist $J' \vartriangleleft R/_I, J' \neq R/_I$ und $J \neq \{I\}$. Also hat $R/_I$ ein echtes Ideal, im Widerspruch dazu, dass $R/_I$ ein Körper ist.
            \item[$\Leftarrow$:] Sei $I$ maximal. Wir behaupten, dass $R/_I$ keine echten Ideale außer dem trivialen hat. Wäre dies nicht so, so sei $J \vartriangleleft R/_I$ echt, $J \neq \{I\}$ und sei $J' := \bigcup_{M \in J} M$. Dann ist $J' \supsetneq I, J' \neq R, J' \vartriangleleft R$, im Widerspruch zur Maximalität von $I$.
        \end{itemize}
        \item Es gilt
        \begin{align*}
            R/_I \text{ ist Integritätsbereich} &\Leftrightarrow (\forall a, b \in R: (a + I)(b + I) = I \Rightarrow a + I = I \lor b + I = I) \\
            &\Leftrightarrow (\forall a, b \in R: ab \in I \Rightarrow a \in I \lor b \in I) \\
            &\Leftrightarrow I \text{ ist prim}.
        \end{align*}
        \item Folgt direkt aus (1) und (2).
        \item Diese Aussage kann leicht mit dem bekannten Lemma von Zorn bewiesen werden.
        Dazu wird die Menge aller echten Ideale $J$ mit $J\supseteq I$ mittels Mengeninklusion partiell geordnet.
        Ist nun $\mathcal{K}$ eine Kette von Idealen, so stellt $\bigcup_{J\in\mathcal{K}}J$ wieder ein Ideal dar.
        Dieses ist tatsächlich echt, denn es gilt für jedes Ideal $J\in\mathcal{K}:1\not\in J$,
        also ist $1\not\in\bigcup_{J\in\mathcal{K}}J$. Klarerweise ist die Vereinigung damit eine obere Schranke und aus dem Lemma
        von Zorn folgt nun die Existenz eines maximalen Elements. Dieses maximale Element ist auch maximal in der Menge
        aller echten Ideale und ist trivialerweise eine Obermenge von $I$.
    \end{enumerate}
\end{proof}

\notedate{04.05.2023}

\begin{example}
    Betrachte den Ring $\mathbb{Z}$, $p \in \mathbb{P}$ und $p \mathbb{Z} \vartriangleleft \mathbb{Z}$, so erhalten wir $\mathbb{Z}/_{p\mathbb{Z}} = \mathbb{Z}_p$.
    $p\mathbb{Z}$ ist dabei ein Primideal und $\mathbb{Z}_p$ ein Körper.

    Für ein $m \in \mathbb{N}\setminus\mathbb{P}$ betrachte $m \mathbb{Z} \vartriangleleft \mathbb{Z}$, so ist $\mathbb{Z} /_{m\mathbb{Z}} = \mathbb{Z}_m$ kein Integritätsbereich,
    also insbesondere kein Körper.
\end{example}

\begin{example}
    Sei $P$ die Menge der Polynomfunktionen auf $\mathbb{R}$ und $(x^2+1) \cdot P \vartriangleleft P$, so ist dies ein Primideal, und $P/_{(x^2+1) \cdot P}$ ein Integritätsbereich. Jedoch ist es kein Körper, da $(x^2+1) \cdot P$ nicht maximal ist -- betrachte dazu beispielsweise
    $$ (x^2 + 1) \cdot P \subsetneq (x^2+1) \cdot P + x \cdot P \vartriangleleft P .$$

    Das Ideal $I := (x^2 - 1) \cdot P \vartriangleleft P$ ist kein Primideal, da $(x-1)\notin I, (x+1) \notin I$, aber $(x-1)(x+1) = x^2 - 1 \in I$.
\end{example}

\begin{definition}
    Wir definieren die \emph{Charakteristik}\index{Charakteristik} eines Rings als
    $$ \chara R := 
    \begin{cases}
        \min\{n\in\mathbb{N}\mid\sum_{i=1}^n 1=0\} & \text{falls existent,}\\
        0 & \text{sonst}.
    \end{cases}
        $$
\end{definition}

\begin{example}
    Für $m\in\mathbb{N}$ ist $\mathbb{Z}_m$ ein bekanntes Beispiel für einen Ring mit Charakteristik $m$.
    $(\mathbb{Z}_m)^{\mathbb{N}}$ ist beispielsweise ein unendlicher Ring mit Charakteristik $m$.
\end{example}

\begin{proposition}\label{prop:frob_ringhomom}
    Sei $R$ ein kommutativer Ring mit 1, $\chara R = p \in \mathbb{P}$ und $k\in\mathbb{N}$. Dann ist
    $$ \varphi : R \to R, x \mapsto x^{p^k} $$
    ein Homomorphismus.
\end{proposition}

\begin{proof}
    Wir zeigen die Aussage mittels Induktion nach $k$.

    Induktionsanfang ($k = 1$): Für $a,b\in R$ gilt
    $$ (a+b)^p = \sum_{i=0}^p {p \choose i} a^i b^{p-i}. $$
    Wir beobachten
    $$ {p \choose i} = \frac{p \cdot (p-1) \cdot \hdots \cdot (p-i+1)}{1\cdot \ldots \cdot i} \equiv 0 \text{ mod } p $$
    für $i\neq 0,p$, daher folgt $(a+b)^p= a^p+b^p$.

    Induktionsschritt ($k\to k+1$): Es gilt in $R$ unmittelbar
    $$(a+b)^{p^{k+1}}=(a^{p^k}+b^{p^k})^p=a^{p^{k+1}}+b^{p^{k+1}}.$$
\end{proof}

\begin{remark}\label{rem:ring-koerper}
    Sei $R$ ein kommutativer Ring mit 1, wir wollen $R$ in einen Körper einbetten. Es gelte $0 \neq 1$ (da sonst für alle $x$ gilt $ x = 1\cdot x = 0 \cdot x = 0$). Wir sammeln nun notwendige Voraussetzungen.

    Es muss $R$ ein Integritätsbereich sein, da $rs=0\land r\neq 0\Rightarrow r^{-1}rs=s=0$ für $r,s\in R$ folgen wird.

    Nicht notwendig (da es aus der vorigen Bedingung folgt), aber interessant ist die Tatsache, dass wenn $R$ ein Integritätsbereich ist, jedes Element außer $0$ bereits kürzbar ist.
    Denn für $r,x,y\in R,r\neq 0$ gilt $rx=ry\Rightarrow r(x-y)=0\Rightarrow x-y=0\Rightarrow x=y$.

    Es ist $R^\times := R \setminus \{0\}$ ein kommutatives Monoid mit der Operation $\cdot$. Definiere eine Äquivalenz\-relation $\sim \subseteq (R \times R^\times)^2, (a, b) \sim (c, d) :\Leftrightarrow ad = bc$. Wie man nachrechnet ist dies sogar eine Kongruenzrelation auf dem multiplikativen Monoid $R\times R^{\times}$. Dann ist $((R \times R^\times)/_\sim, \cdot, [(1,1)]_\sim) =: M$ ein Monoid, wobei jedes $[(x,y)]_\sim$ mit $x \neq 0$ ein Inverses besitzt.
    
    Dann ist
    $$ \varphi : R \to M, x \mapsto [(x, 1)]_\sim $$
    eine Einbettung von $R$ als multiplikatives Monoid in $M$.

    Für jedes multiplikative Monoid $N$ mit einer Einbettung $\psi : R \to N$ und der Eigenschaft
    $$\forall x\in R^\times\exists y\in N:y\psi(x)=\psi(x)y=1$$
    gibt es eine Einbettung $\bar{\psi} : M \to N$ mit $\bar{\psi} \circ \varphi = \psi$.

    Auf $(R\times R^\times)$ definieren wir nun eine Addition
    $$ (a,b) + (c,d) := (ad+bc, bd), \quad -(a,b) := (-a, b), $$
    so ist $(R \times R^\times, +, (0,1), -)$ eine Gruppe.
\end{remark}

\begin{lemma}
    Die in \cref{rem:ring-koerper} definierte Äquivalenzrelation $\sim \subset (R \times R^\times)^2$ ist eine Kongruenzrelation bezüglich $+$.
\end{lemma}

\begin{proof}
    Seien $(z_1,n_1),(z_1',n_1'),(z_2,n_2),(z_2',n_2')\in R\times R^\times$ mit $(z_1,n_1)\sim (z_1',n_1')$ und $(z_2,n_2)\sim (z_2',n_2')$ gegeben.
    Dann ist zu zeigen, dass $(z_1n_2+z_2n_1,n_1n_2)\sim (z_1'n_2'+z_2'n_1',n_1'n_2')$ gilt. Die Behauptung folgt durch Einsetzen
    in die Definition:
    $$(z_1n_2+z_2n_1)n_1'n_2'=\overbrace{z_1n_1'}^{z_1'n_1}n_2n_2'+\overbrace{z_2n_2'}^{z_2'n_2}n_1n_1'=(z_1'n_2'+z_2'n_1')n_1n_2$$
\end{proof}

\begin{theorem}\label{theorem:Quotientenkoerper}
    Sei $R$ ein Integritätsbereich mit $1 \neq 0$ und sei weiters $\sim \subseteq (R \times R^\times)^2$ wie in \cref{rem:ring-koerper} definiert. Dann gilt:
    \begin{enumerate}
        \item $K := (R \times R^\times)/_\sim $ mit $+, \cdot$ aus \cref{rem:ring-koerper} ist ein Körper.
        \item $\varphi : R \to K, x \mapsto [(x, 1)]_\sim$ ist eine Einbettung.
        \item Für alle Einbettungen $\psi : R \to L$ in einen Körper $L$ gibt es eine Einbettung $\bar{\psi} : K \to L$ mit $\bar{\psi} \circ \varphi = \psi$.
    \end{enumerate}
\end{theorem}

\begin{proof}
    Wir haben bereits gezeigt, dass $\sim$ eine Kongruenzrelation ist, womit $K$ wohldefiniert ist.

    Wir wissen $(K, \cdot, [(1,1)]_\sim)$ ist ein kommutatives Monoid.

    Weiters ist $(K \setminus \{ [(0,1)]_\sim \}, \cdot, [(1, 1)]_\sim) = (R^\times \times R^\times)/_\sim$ eine kommutative Gruppe, genauso auch $(K, +, [(0,1)]_\sim, -)$.

    Das Distributivgesetz verifiziert man unmittelbar durch Nachrechnen.

    Nach Konstruktion ist $\varphi$ eine injektive Einbettung bezüglich $\cdot$. Allerdings gilt
    für $a,b\in R$, dass $\varphi(a+b)=[(a+b,1)]_\sim=[(1a+1b,1\cdot 1)]_\sim=[(a,1)]_\sim+[(b,1)]_\sim=\varphi(a)+\varphi(b)$.
    Wegen $\varphi(0)=[(0,1)]_\sim=0$ wird auch das neutrale Element von $\varphi$ erhalten,
    woraus bereits die Verträglichkeit mit additiven Inversen folgt. Daher ist $\varphi$ auch eine Einbettung bezüglich $+$.

    Sei $\psi : R \to L$ eine Einbettung in einen Körper $L$. Nach der Monoidkonstruktion gibt es eine Einbettung $\bar{\psi} : K \to L$ bezüglich $\cdot$ mit $\bar{\psi} \circ \varphi = \psi$.
    Wir verifizieren nun, dass $\bar{\psi}$ mit der Addition verträglich ist:
    \begin{align*}
        \bar{\psi} ([(z_1, n_1)]_\sim + [(z_2, n_2)]_\sim) &= \bar{\psi} ([(z_1 n_2 + z_2 n_1, n_1 n_2)]_\sim) \\
        &= \bar{\psi}([(z_1n_2+z_2n_1,1)]_\sim)\cdot \bar{\psi}([(1,n_1n_2)]_\sim)\\
        &= \overbrace{\bar{\psi}\circ\varphi}^{=\psi}(z_1n_2+z_2n_1)\cdot \bar{\psi}([(1,n_1n_2)]_\sim)\\
        &= \big(\overbrace{\psi}^{\bar{\psi}\circ\varphi}(z_1n_2)+\overbrace{\psi}^{\bar{\psi}\circ\varphi}(z_2n_1)\big)\cdot \bar{\psi}([(1,n_1n_2)]_\sim)\\
        &= \bar{\psi}([(z_1n_2,1)]_\sim)\cdot \bar{\psi}([(1,n_1n_2)]_\sim)+\bar{\psi}([(z_2n_1,1)]_\sim)\cdot \bar{\psi}([(1,n_1n_2)]_\sim)\\
        &= \bar{\psi}([(z_1,n_1)]_\sim)+\bar{\psi}([(z_2,n_2)]_\sim)
    \end{align*}
\end{proof}

\begin{proposition}
    Sei $K'$ ein Körper mit Eigenschaft (3) aus \cref{theorem:Quotientenkoerper}, so gilt bereits $K' \cong K$, wobei $K$ unser konstruierter Körper aus \cref{theorem:Quotientenkoerper} (1) ist.
\end{proposition}

\begin{proof}
    Gegeben sind also $\varphi:R\to K$ und $\varphi_0:R\to K'$ jeweils mit Eigenschaft (3). Für $K'$ bedeutet das: Für jeden Körper $L$ und jede Ringeinbettung $\psi:R\to L$ gibt es eine Körpereinbettung $\bar{\psi}:K'\to L$ mit $\bar{\psi}\circ\varphi_0=\psi$.
    Daher existieren $\bar{\varphi}:K'\to K$ mit $\bar{\varphi}\circ \varphi_0=\varphi$ und $\bar{\varphi_0}:K\to K'$ mit $\bar{\varphi_0}\circ \varphi=\varphi_0$. Wir zeigen zuerst die folgende Behauptung: Für jeden injektiven Homomorphismus
    $\xi:K\to K$ mit $\xi|_{\varphi(R)}=\id_R$ folgt $\xi=\id$. Dies folgt aus der folgenden Rechnung:
    $$
        \xi([(a,b)]_\sim)=\xi([(a,1)]_\sim)\xi([1,b]_\sim)=[(a,1)]_\sim\cdot [(b,1)]_\sim^{-1}=[(a,b)]_\sim.
    $$
    Insbesondere gilt daher $\bar{\varphi}\circ \bar{\varphi_0}=\mathrm{id}_K$, da $(\bar{\varphi}\circ\bar{\varphi_0}\circ\varphi)(a)=(\bar{\varphi}\circ\varphi_0)(a)=\varphi(a)$ gilt. Damit ist $\bar{\varphi}$ ein Isomorphismus von $K'$ nach $K$.
\end{proof}

\begin{definition}
    Sei $R$ ein kommutativer Ring mit $1\neq 0$ und ein Integritätsbereich. Dann wird der Körper
    $(R\times R^\times)/_\sim$ aus Satz \ref{theorem:Quotientenkoerper} \emph{Quotientenk\"orper von $R$}\index{Quotientenkörper} genannt.
    Für $[(z,n)]_\sim$ schreibt man auch $\frac{z}{n}$.
\end{definition}

\begin{remark}
    Die letzten beiden Theoreme liefern uns folgendes Ergebnis: Zu einem Integritätsbereich mit $1\neq 0$, kann der Quotientenk\"orper konstruiert werden. Dieser ist (bis auf Isomorphie) eindeutig bestimmt und der kleinste Körper
    der $R$ enthält.
\end{remark}

\begin{definition}\label{def:polynomring}
    Sei $R$ ein Ring mit $1$. Wir definieren den \emph{Polynomring über $R$}\index{Polynomring}
    $$R[x]:=\left\{(x_n)_{n\in \mathbb{N}} \in R^\mathbb{N} \;\middle|\; \vert\{x_n\neq 0\mid n\in \mathbb{N}\}\vert<\infty\right\}$$
    mit den Operationen
    \begin{align*}
        + &: R[x] \times R[x] \to R[x], ((x_n)_{n\in \mathbb{N}}, (y_n)_{n\in\mathbb{N}}) \mapsto (x_n+y_n)_{n\in\mathbb{N}}\\
        \cdot &: R[x] \times R[x] \to R[x], ((x_n)_{n\in \mathbb{N}}, (y_n)_{n\in\mathbb{N}}) \mapsto \left(\sum_{i=0}^n x_i y_{n-i}\right)_{n\in\mathbb{N}}
    \end{align*}

    Elemente von $R[x]$ bezeichnen wir als \emph{Polynome}\index{Polynom} und die einzelnen Folgenglieder der Elemente als \emph{Koeffizienten}\index{Koeffizient}. Außerdem definieren wir für $(x_n)_{n\in\mathbb{N}}\in R[x]$ den \emph{Grad}\index{Grad}:
    $$\deg ((x_n)_{n\in\mathbb{N}}):=
    \begin{cases}
        -\infty, & (x_n)_{n\in\mathbb{N}}=(0)_{n\in\mathbb{N}}\\
        \max\{n\in\mathbb{N}:x_n\neq 0\}, & \text{sonst.}
    \end{cases}$$

    Weiter definieren wir den Ring der \emph{formalen Potenzreihen}\index{formale Potenzreihe} 
    $$R[[x]]:=\left\{(x_n)_{n \in \mathbb{N}} \in R^\mathbb{N}\right\}$$
    mit denselben Operation wie oben. 
    
    Die Elemente von $R[x]$ wollen wir auch als $p(x)=\sum_{i=0}^na_ix^i$ auffassen, wenn $a_i=0$ für $i>n$ gilt. Formal ist hier eigentlich die Folge
    der Koeffizienten ein Element des Ringes. Weiters schreiben wir für die Elemente von $R[[x]]$ auch
    $p(x)=\sum_{i=0}^\infty a_ix^i$.
\end{definition}

\begin{remark}
    Alternativ kann der Polynomring wie folgt definiert werden:

    Sei $R$ ein kommutativer Ring mit 1, $x$ eine Variable und definiere
    $$ R[x] := \{ t(x) \mid t \text{ Term über $x$ in Sprache } +, \cdot, (r)_{r \in R} \}/_\sim, $$
    wobei $\sim$ die von Gesetzen der kommutativen Ringe mit 1 und Gesetzen in $R$\footnote{Gemeint sind hierbei Gesetze wie beispielsweise $r\cdot(s\cdot x)=(r\cdot s)\cdot x$ oder $(r\cdot x)+(s\cdot x)=(r+s)\cdot x$, also Gleichungen welche in $R$ gelten.} erzeugte Äquivalenz\-relation ist. In $R[x]$ gilt also beispielsweise
    $$ x + x \cdot x = x \cdot x + x, \quad r \cdot (s \cdot x) = (r \cdot s) \cdot x. $$

    Vorteil von \cref{def:polynomring} ist, dass analog auch die Verallgemeinerung der formalen Potenzreihen definiert werden kann, was mit diesem Ansatz nicht möglich ist.
\end{remark}

Folgende Punkte sind einfach nachzurechnen:

\begin{proposition}\label{prop:polynomringe}
    Sei $R$ ein kommutativer Ring mit $1$. Dann gilt:
    \begin{itemize}
        \item $R[{x}]$ ist ein kommutativer Ring mit $1$.
        \item $R[x]\le R[[x]]$
        \item $R$ ist in $R[x]$ eingebettet vermöge $r\mapsto \sum_{i=0}^0rx^i$
        \item $R$ ist ein Integritätsbereich $\Leftrightarrow$ $R[x]$ und $R[[x]]$ sind Integritätsbereiche.
    \end{itemize}
\end{proposition}

\notedate{10.05.2023}
\begin{definition}
    Sei $R$ ein Integritätsbereich mit $1\neq 0$. Dann nennen wir den Quotientenkörper von $R[x]$ 
    $$R(x):=\left\{\frac{p(x)}{q(x)}\;\middle|\;p(x),q(x)\in R[x],q(x)\neq 0\middle\}\right/_\sim$$
    mit der üblichen Relation $\frac{p}{q}\sim \frac{r}{s}\Leftrightarrow sp=qr$ den Körper der \emph{gebrochen rationalen Funktionen}\index{gebrochen rationale Funktion}.
\end{definition}

\begin{remark}
    Ist $R$ ein Integritätsbereich mit $1\neq 0$, so kann der Quotientenk\"orper $K$ und dann von diesem
    der Polynomring $K[x]$ betrachten werden. Dieser besitzt nun einen Quotientenk\"orper $K(x)$. Andererseits
    kann man auch den Quotientenk\"orper des Polynomrings über $R$ betrachten und erhält durch $R(x)$ einen dazu isomorphen Körper,
    also $K(x)\cong R(x)$.
\end{remark}

\begin{remark}
    Als Verallgemeinerung des Polynomrings kann man auch den Polynomring in $n$ Variablen $x_1,\ldots,x_n$ rekursiv definieren
    durch $R[x_1,\ldots,x_n]:=R[x_1,\ldots,x_{n-1}][x_n]$. Auch für eine beliebige Variablenmenge $X$ kann eine Verallgemeinerung
    getroffen werden, indem man mit $R[x]$ die Terme über der Sprache $(+,0,-,\cdot,1,(x)_{x\in X},(m_r)_{r\in R})$ nach den Ringgesetzen
    und Gleichheiten in $R$ faktorisiert.
\end{remark}

\begin{definition}
    Sei $K$ ein Körper. Dann heißt $K$ \emph{algebraisch abgeschlossen}\index{Körper!algebraisch abgeschlossen}, wenn
    $$\forall p(x)\in K[x]:p(x)\not\in\footnote{Hier wird $K$ mittels der Einbettung aus \cref{prop:polynomringe} als Teilmenge betrachtet.} K\Rightarrow \exists a\in K:p(a)\footnote{Der Ausdruck p(a) wird mithilfe des Einsetzungshomomorphismus definiert.}=0$$
    gilt.
\end{definition}

\begin{theorem}[Nullstellensatz von Hilbert, klein]\label{theorem:nullstellensatz}\index{Nullstellensatz von Hilbert}
    Sei $K$ ein algebraisch abgeschlossener Körper und $I\vartriangleleft K[x_1,\ldots,x_n]$ ein echtes Ideal.
    Dann gilt $\exists(a_1,\ldots,a_n)\in K^n:\forall p(x_1,\ldots,x_n)\in I:p(a_1,\ldots,a_n)=0.$
\end{theorem}

\begin{remark}
    Satz \ref{theorem:nullstellensatz} ist nicht Teil dieser Lehrveranstaltung, sondern soll nur einen Ausblick auf die Algebra 2 Vorlesung geben.
    Die Anforderung an ein echtes Ideal sind dabei sehr natürlich. Ein echtes Ideal kann nämlich keine Konstanten $c$ enthalten, da sonst
    $c\in I\Rightarrow c^{-1}c\in I\Rightarrow I=K[x_1,\ldots,x_n]$ gilt. Ist außerdem $F$ eine beliebige Menge von
    Polynomen mit einer gemeinsamen Nullstelle, so überzeugt man sich leicht davon, dass auch jedes Polynom aus dem erzeugten Ideal
    von $F$ an dieser Stelle den Wert $0$ annimmt. Daher kann \obda angenommen werden, dass $F$ sogar ein Ideal ist.
\end{remark}

\begin{proposition}
    Sei $R$ ein kommutativer Ring mit $1$ und $X\neq\emptyset$ eine Variablenmenge. Dann gilt:
    \begin{enumerate}
        \item $R\leq R[X]$
        \item Für jeden Ring $S$ mit $R\leq S$ und jeden Homomorphismus $\varphi:X\to S$
        existiert genau ein Homomorphismus $\bar{\varphi}:R[X]\to S$, sodass $\bar{\varphi}|_X=\varphi$ und $\bar{\varphi}|_R=\id_R$ gilt.
    \end{enumerate}
\end{proposition}

\begin{proof}
    Der Beweis verläuft analog wie bei \cref{theorem:variablenmenge_frei}.
\end{proof}


\begin{definition}
    Sei $R$ ein Ring und $I\vartriangleleft R$. Dann definieren wir für $r,s\in R$:
    $$r\equiv s\mod I:\Leftrightarrow r-s\in I.$$
    Wir sagen auch \emph{$r$ ist $s$ modulo $I$}\index{modulo}.
\end{definition}

\begin{theorem}[Chinesischer Restsatz, allgemein]\index{Chinesischer Restsatz!allgemein}
    Seien $R$ ein kommutativer Ring mit 1 und $I_1,\ldots,I_n\vartriangleleft R$ mit $\forall i\neq j\Rightarrow I_i+I_j=R$.
    
    Dann wird $I:=\bigcap_{i=1}^nI_i$ definiert und es gilt:
    \begin{enumerate}
        \item $\forall r_1\ldots,r_n\in R\,\exists r\in R:\forall i\in\{1,\ldots,n\}:r\equiv r_i\mod I_i$. Weiters ist $r$
        modulo $I$ eindeutig bestimmt.
        \item $\varphi:R/_I\to R/_{I_1}\times\ldots\times R/_{I_n}, r+I\mapsto (r+I_1,\ldots,r+I_n)$ ist ein Isomorphismus.
    \end{enumerate}
\end{theorem}

\begin{proof}
    Zuerst stellen wir die Behauptung $\forall i=2,\ldots,n: I_1+(I_2\cap \ldots \cap I_n)=R$ auf, welche wir mit Induktion
    beweisen wollen:

    Induktionsanfang $(i=2)$: Die Behauptung gilt laut Voraussetzung.

    Induktionsschritt $(i\to i+1)$: Da $R$ ein Ring mit $1$ ist gilt
    $R=R\cdot R$. Nun kann die Induktionsannahme auf den ersten Faktor und die Voraussetzung des Satzes auf den zweiten Faktor angewendet werden,
    woraus man $R\cdot R=(I_1+(I_2\cap\ldots\cap I_i))\cdot(I_1+I_{i+1})$ erhält. Das ist offensichtlich eine Teilmenge
    von $I_1+(I_2\cap\ldots\cap I_i)\cdot I_{i+1}$. Der zweite Summand ist eine Teilmenge von $I_{i+1}$, da
    $I_{i+1}$ ein (Links-)Ideal ist. Gleichzeitig ist er eine Teilmenge von $I_2\cap\ldots \cap I_i$, da diese Menge ein (Rechts-)Ideal ist.
    Damit folgt, dass $R=R\cdot R$ schon in $I_1+(I_2\cap\ldots \cap I_{i+1})$ enthalten sein muss, also die Gleichheit.

    Analog gilt mit der Definition $I_i':=\bigcap_{j\neq i}I_j$, dass für alle $i\in\{1,\ldots,n\}$
    auch $I_i+I_i'=R$ ist. Daher existieren für jedes $i\in\{1,\ldots,n\}$ ein $a_i\in I_i$ und ein $a_i'\in I_i'$
    mit $r_i=a_i+a_i'$. Definiert man nun $r:=\sum_{i=1}^na_i'$, so erhält man für alle $i\in\{1,\ldots,n\}$, dass
    $r\equiv a_i'\equiv r_i \text{ mod }I_i$ gilt, also die Existenz.

    Dieses Element ist eindeutig modulo $I$ bestimmt, denn falls $r'$ und $r$ beide die gewünschte Eigenschaft haben,
    so folgt $r'-r\in I_i$ für alle $i$, also $r'-r\in\bigcap_{i=1}^nI_i=I$.

    Schließlich ist die Abbildung $\varphi$ laut Definition wohldefiniert. Die Surjektivität ist die Existenz von $r$ im ersten Punkt,
    die Injektivität ist die Eindeutigkeit modulo $I$. Für die Homomorphiebedingung rechnen wir exemplarisch nach, dass
    $\varphi$ mit der Addition verträglich ist:
    $$\varphi((r+I)+(s+I))=\varphi((r+s)+I)=((r+s)+I_1,\ldots,(r+s)+I_n)=\varphi(r+I)+\varphi(s+I).$$
    Die Multiplikation zeigt man analog.
\end{proof}

Aus dem obigen Satz folgt unmittelbar:

\begin{corollary}[Chinesischer Restsatz, klassisch]\index{Chinesischer Restsatz!klassisch}
    Seien $m_1,\ldots,m_n\geq 2$ und $\forall i\neq j: m_i\mathbb{Z}+m_j\mathbb{Z}=\mathbb{Z}$
    oder äquivalent dazu $\text{ggT}(m_i,m_j)=1$. Dann gilt
    \begin{enumerate}
        \item $\forall a_1,\ldots,a_n\in\mathbb{Z}\,\exists a\in\mathbb{Z}:\forall i\in\{1,\ldots,n\}: a\equiv a_i \mod m_i$.
        Weiters ist dieses $a$ eindeutig modulo $\bigcap_{i=1}^{n}m_i\mathbb{Z}=m_1\ldots m_n\mathbb{Z}$.
        \item $\varphi:\mathbb{Z}_{m_1\ldots m_n}\to \mathbb{Z}_{m_1}\times \mathbb{Z}_{m_n}, [a]\mapsto (a\mod m_1,\ldots, a\mod m_n)$
        ist ein Isomorphismus.
    \end{enumerate}
\end{corollary}


